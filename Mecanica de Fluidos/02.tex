\section{Estática de fluidos}

\subsection{Derivada y series de Taylor}

La pendiente de una recta con base de una función
\[
    \begin{split}
        m & = \frac{ y_{2} - y_{1} }{ x_{2} - x_{1} } \\
        & = \frac{ f(x_{2}) - f(x_{1}) }{ x_{2} - x_{1} } \\
        & = \frac{ f(x_{1} + h ) - f(x_{1}) }{ h }
    \end{split}
\]

La pendiente de la recta tangente en el punto x, o derivada ocurre cuando
\[
    \frac{ df(x) }{ dx } = \lim_{h \to 0} \frac{ f(x_{1} + h ) - f(x_{1}) }{ h }
\]

Los coeficientes de \( (a + b)^{n} \) se pueden escribir como 
\[
    1, \frac{ n }{ 1 }, \frac{ n(n-1) }{ 1 \cdot 2 }, \frac{ n(n-1)(n-2)) }{ 1 \cdot 2 \cdot 3 }, \cdots
\]

El cambio de la altura de una función se puede escribir como
\[
    \Delta f(x) = f(x + r) - f(x)
\]

El valor de la altura final es
\[
    f(x + r) = f(x) + \Delta f(x)
\]

Se aplica lo misma operación una segunda vez
\[
    \begin{split}
        \Delta^{2} f(x) & = \Delta (\Delta f(x)) \\
        & = \Delta( f(x + r) -f(x) ) \\
        & = f(x + r + r) - f(x + r) - \Delta f(x) , \; \text{se sustituye \( f(x + r) \)}\\
        & = f(x + 2r) - 2\Delta f(x) - f(x)\\
    \end{split}
\]

El valor de la altura final es
\[
    f(x + 2r) = f(x) + 2\Delta f(x) + \Delta^{2}f(x)
\]

Se vuelve a repetir 
\[
    \begin{split}
        \Delta^{3} f(x) & = \Delta (\Delta^{2} f(x)) \\
        & = f(x + 3r) - f(x + 2r) - 2\Delta^{2} f(x) - \Delta f(x), \; \text{se sustituye \( f(x + 2r) \)} \\
        & = f(x + 3r) - f(x) - 2\Delta f(x) - \Delta^{2}f(x) - 2\Delta f(x) - \Delta f(x) \\
        & = f(x +3r) - f(x) - 3\Delta f(x) - 3\Delta^{2} f(x)
    \end{split}
\]

El valor de la altura es
\[
    f(x + 3r) = f(x) + 3\Delta f(x) + 3\Delta^{2} f(x) + \Delta^{3}f(x)
\]

Por lo tanto
\[
    f(x + nr) = f(x) + \frac{ n }{ 1 } \Delta f(x) + \frac{ n(n - 1) }{ 1 \cdot 2 } \Delta^{2} f(x) + \cdots + \frac{ n(n - 1) \cdots 1 }{ 1 \cdot 2 \cdots n } \Delta^{n} f(x)
\]

Haciendo \( h = nr \), y multiplicando por 1
\[
    f(x + nr) = f(x) + \frac{ nr }{ 1 } \frac{ \Delta f(x) }{ r } + \frac{ n(n - 1) r^{2} }{ 1 \cdot 2 } \frac{ \Delta^{2} f(x) }{ r^{2} }  + \cdots + \frac{ ( n(n - 1) \cdots 1 ) r^{n} }{ 1 \cdot 2 \cdots n } \frac{ \Delta^{n} f(x) }{ r^{n} }
\]

Haciendo \( n \to \infty \) hace que \( r \to 0 \)
\[
    f(x + rn) = f(x + h) = f(x) + hf(x)^{'} + h^{2} \frac{ f(x)^{''} }{ 2! } + \cdots
\]

\subsection{Principio de Pascal}
presión 
\[
    P = \frac{ F }{ A } \; \Big[ 1Pa = {1}^{N}/_{m^{2}}, {}^{lb}/_{pie^{2}}, {}^{lb}/_{pulg^{2}} \Big]
\]

Presión atmosférica 
\[
    Pa \;\;
    \Big[ 1 \; atm = 760 \; mmHg = 16.7 {}^{lb}/_{pulg^{2}} = 101.325kPa \Big]
\]

Presión relativa
\[
    P_{rel} 
\]

Presión absoluta
\[
    P_{abs} = P_{atm} \pm P_{rel}
\]

%Figura

\[
    \begin{split}
        & \sum F_{y} = ma_{y} \\
        & P_{y} \Delta z \Delta x - P \Delta s \Delta x \sin{ \theta } = \rho \frac{ \Delta z \Delta y \Delta x }{ 2 } a_{y} \\
        & \sum F_{z} = ma_{z} \\
        & P_{z} \Delta y \Delta x - P \Delta s \Delta x \cos{ \theta } - \rho g \frac{ \Delta z \Delta y \Delta x }{ 2 } = \rho \frac{ \Delta z \Delta y \Delta x }{ 2 } a_{z}
    \end{split}
\]

Pero de la figura se puede ver que
\[
    \begin{split}
        \Delta z = \Delta s \sin{ \theta } \\
        \Delta y = \Delta s \cos{ \theta }
    \end{split}
\]

entonces 
\[
    \begin{split}
        P_{y} - P = \rho \frac{ \Delta y }{ 2 } a_{y} \\
        P_{z} - P = \frac{ \rho }{ 2 } (g + a_{z}) \Delta z
    \end{split}
\]

Si \( \Delta y \to 0 \), \( \Delta z \to 0 \), y \( \frac{ d \Vec{v} }{ dy } = 0 \)
\[
    \begin{split}
        p_{y} - P & = 0 \\
        P_{z} - P & = 0 \\
        P = P_{z} & = P_{y}
    \end{split}
\]

Transmisor de fuerza. Si la presión de un punto es igual a otro
\[
    \begin{split}
        P_{1} & = P_{2} \\
        \frac{ F_{1} }{ A_{1} } & = \frac{ F_{2} }{ A_{2} }
    \end{split}
\]

%figura

Transmisor de presión
\[
    P_{1} A_{1} = P_{2} A_{2}
\]

%figura

si \( \Delta z = 1 \)
\[
    P = P_{x} = P_{y} = P_{z}
\]

\subsection{Variación de la presión de un fluido en reposo}

%Figura

\[
    \begin{split}
        W & = dm g \\
        & = \rho g d \forall \\
        & = \rho g dx dy dz
    \end{split}
\]

\[
    \begin{split}
        f(x + \Delta x) & = F(X) + \frac{ \partial f(x) }{ \partial x } \cdot \Delta x + \frac{ \partial^{2} f(x) }{ \partial x^{2} } \cdot \frac{ \Delta x }{ 2! } + \cdots \\
        f(x + \delta x) & \approx f(x) + \frac{ \partial f(x) }{ \partial x } \Delta x \\
        f(x - \delta x) & \approx f(x) - \frac{ \partial f(x) }{ \partial x } \Delta x \\
    \end{split}
\]

%Figura

\[
    \begin{split}
        (CD) \; P(y + \frac{ dy }{ 2 }) & = P + \frac{ \partial P }{ \partial y } \frac{ dy }{ 2 } \\
        (CI) \; P(y - \frac{ dy }{ 2 }) & = P - \frac{ \partial P }{ \partial y } \frac{ dy }{ 2 }
     \end{split}
\]

\[
    \begin{split}
        dFsy & = dfy (+) - dfy(-) \\
        & = [ ( P + \frac{ \partial P }{ \partial y } \frac{ dy }{ 2 } ) - ( P - \frac{ \partial P }{ \partial y } \frac{ dy }{ 2 } ) ] dxdz \\
        & = - \frac{ \partial P }{ \partial y } dx dy dz \;\; (1) \\
        Fsz & = - \frac{ \partial P }{ \partial z } dx dy dz \;\; (2) \\
        Fsx & = - \frac{ \partial P }{ \partial x } dx dy dz \;\; (3) \\
        \underbrace{ dF_{B} }_{ \text{cuerpo o másicas} } & = \rho g dx dy dz \;\; (4)
    \end{split}
\]

Fuerza másica: fuerza que actúa sobre la masa de un fluido.

\[
    \begin{split}
        d\Vec{ F_{T} } & = m\Vec{ a } \\
        d\Vec{ F_{s} } + d\Vec{ F_{B} } & = m\Vec{ a } \\
        d\Vec{ F_{sx} } +d\Vec{ F_{sy} } + d\Vec{ F_{sy} } + \rho g dx dy dz & = m\Vec{ a } \\
        ( -\frac{ \partial P }{ \partial z } -\frac{ \partial P }{ \partial y } -\frac{ \partial P }{ \partial x } + \rho g ) dx dy dz & = \rho dx dy dz \Vec{ a }
    \end{split}
\]

\[
    \underbrace{ -\Vec{ \nabla } P + \rho \Vec{ g } = \rho \Vec{ a }}_{ \text{Ecuación de cantidad de movimiento} }
\]

\[
    \underbrace{ -\Vec{ \nabla } P + \rho \Vec{ g } = 0 }_{ \text{Ecuación vectorial de hidrostática} }
\]

\[
    \begin{split}
        - \frac{ \partial P }{ \partial x } + \rho g_{x} & = 0 \\
        - \frac{ \partial P }{ \partial y } + \rho g_{y} & = 0 \\
        - \frac{ \partial P }{ \partial z } + \rho g_{z} & = 0 \\
    \end{split}
\]

Si \( g_{x} = g_{y} = 0 \) y \( g_{z} = -g \)

\[
    \begin{split}
        -\frac{ \partial P }{ \partial z } = \rho g \\
        \frac{ \partial P }{ \partial z } = -\rho g \\
        \partial P = -\rho g \partial z \\
        \int_{ P }^{ 0 } dP = - \rho g \int_{ -h }^{ 0 } dz \\
        - P = -\rho g (+h) \\
        P = \rho g h
    \end{split}
\]

\subsection{ Fuerzas sobre superficies planas y curvas }

%figura

\subsubsection{Fuerza resultante de una distribución de fuerzas hidrostáticas}

considerando
\[
    \begin{split}
        dF_{R} & = PdA \\
        \frac{ dP }{ dh } & = \rho g \\
        P & = P_{o} + \rho g h \\
        h & = y \sin{ \theta }
    \end{split}
\]

\[
    \begin{split}
        dF_{R} & = ( P_{o} + \rho g h ) dA \;\; P_{o} = 0 \\
        & = \rho g h dA \\
        \int_{0}^{F_{R}} & = \rho g \sin{ \theta } \int y dA \\ 
        F_{R} & = \rho g \sin{ \theta } y_{cg} A \\
    \end{split}
\]

\[
    \underbrace{ \int_{A} y_{c} dA = y_{cg} A }_{\text{Primer momento de área}} 
\]

\subsubsection{Localización de la fuerza resultante}
\[
    \begin{split}
        dM_{x} & = dF_{R} \cdot y \\
        & = \rho g y \sin{ \theta } y dA \\
        & = \rho g y^{2} \sin{ \theta } dA \\
        M_{Rx} & = \rho g \sin{ \theta } I_{xx} \\
        F_{R} y_{cp} & = \rho g \sin{ \theta } I_{xx} \\
        y_{cp} & = \frac{ \rho g \sin{ \theta } I_{xx} }{ \rho g \sin{ \theta } y_{cg} A } = \frac{ I_{cg} + y_{cg}^{2} A }{ y_{cg} A } = \frac{ I_{cg} }{ y_{cg} A } + y_{cg} \\
    \end{split}
\]

\[
    \underbrace{ \int y^{2} dA = I_{xx} = I_{cg} + y_{cg}^{2} A }_{ \text{Segundo momento de área} }
\] 

%Figuras 1, 2, 3, 4

Ejemplo

%figura

\[
    \begin{split}
        F_{H} & = F_{x2} - F_{x1} = 0 \\
        F_{x2} & = F_{x1} \\
        F_{V} & = 2 F_{y} + 2W_{L}
    \end{split}
\]

\[
    \begin{split}
        F_{x} & = \rho g (h_{1} + \frac{ r }{ 2 }) (rb) \\
        F_{y} & = \rho g h_{1} (rb) \\
        W & = mg \\
        & = \rho \forall g = \rho g (R^{2}b - \frac{ \pi r^{2} b }{ 4 })
    \end{split}
\]

\subsection{Principio de Arquímedes}
%figura

\[
    \begin{split}
        F_{B} & = F_{i} - F_{s} \\
        & = \rho g ( h + y ) A - \rho g h A \\
        & = \rho g y A \\
        & = \rho g \forall
    \end{split}
\]

\[
    F_{B}: \; \text{Fuerza de flotabilidad o Boyante}
\]

\[
    \forall: \; \text{ Volumen desplazado}
\]