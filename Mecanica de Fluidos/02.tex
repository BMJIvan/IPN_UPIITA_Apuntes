\section{Estática de fluidos}

\subsection{Derivada y series de Taylor}

La pendiente de una recta con base de una función
\[
    \begin{split}
        m & = \frac{ y_{2} - y_{1} }{ x_{2} - x_{1} } \\
        & = \frac{ f(x_{2}) - f(x_{1}) }{ x_{2} - x_{1} } \\
        & = \frac{ f(x_{1} + h ) - f(x_{1}) }{ h }
    \end{split}
\]

La pendiente de la recta tangente en el punto x, o derivada ocurre cuando
\[
    \frac{ df(x) }{ dx } = \lim_{h \to 0} \frac{ f(x_{1} + h ) - f(x_{1}) }{ h }
\]

Los coeficientes de \( (a + b)^{n} \) se pueden escribir como 
\[
    1, \frac{ n }{ 1 }, \frac{ n(n-1) }{ 1 \cdot 2 }, \frac{ n(n-1)(n-2)) }{ 1 \cdot 2 \cdot 3 }, \cdots
\]

El cambio de la altura de una función se puede escribir como
\[
    \Delta f(x) = f(x + r) - f(x)
\]

El valor de la altura final es
\[
    f(x + r) = f(x) + \Delta f(x)
\]

Se aplica lo misma operación una segunda vez
\[
    \begin{split}
        \Delta^{2} f(x) & = \Delta (\Delta f(x)) \\
        & = \Delta( f(x + r) -f(x) ) \\
        & = f(x + r + r) - f(x + r) - \Delta f(x) , \; \text{se sustituye \( f(x + r) \)}\\
        & = f(x + 2r) - 2\Delta f(x) - f(x)\\
    \end{split}
\]

El valor de la altura final es
\[
    f(x + 2r) = f(x) + 2\Delta f(x) + \Delta^{2}f(x)
\]

Se vuelve a repetir 
\[
    \begin{split}
        \Delta^{3} f(x) & = \Delta (\Delta^{2} f(x)) \\
        & = f(x + 3r) - f(x + 2r) - 2\Delta^{2} f(x) - \Delta f(x), \; \text{se sustituye \( f(x + 2r) \)} \\
        & = f(x + 3r) - f(x) - 2\Delta f(x) - \Delta^{2}f(x) - 2\Delta f(x) - \Delta f(x) \\
        & = f(x +3r) - f(x) - 3\Delta f(x) - 3\Delta^{2} f(x)
    \end{split}
\]

El valor de la altura es
\[
    f(x + 3r) = f(x) + 3\Delta f(x) + 3\Delta^{2} f(x) + \Delta^{3}f(x)
\]

Por lo tanto
\[
    f(x + nr) = f(x) + \frac{ n }{ 1 } \Delta f(x) + \frac{ n(n - 1) }{ 1 \cdot 2 } \Delta^{2} f(x) + \cdots + \frac{ n(n - 1) \cdots 1 }{ 1 \cdot 2 \cdots n } \Delta^{n} f(x)
\]

Haciendo \( h = nr \), y multiplicando por 1
\[
    f(x + nr) = f(x) + \frac{ nr }{ 1 } \frac{ \Delta f(x) }{ r } + \frac{ n(n - 1) r^{2} }{ 1 \cdot 2 } \frac{ \Delta^{2} f(x) }{ r^{2} }  + \cdots + \frac{ ( n(n - 1) \cdots 1 ) r^{n} }{ 1 \cdot 2 \cdots n } \frac{ \Delta^{n} f(x) }{ r^{n} }
\]

Haciendo \( n \to \infty \) hace que \( r \to 0 \)
\[
    f(x + rn) = f(x + h) = f(x) + hf(x)^{'} + h^{2} \frac{ f(x)^{''} }{ 2! } + \cdots
\]

\subsection{Principio de Pascal}
presión 
\[
    P = \frac{ F }{ A } \; \Big[ 1Pa = {1}^{N}/_{m^{2}}, {}^{lb}/_{pie^{2}}, {}^{lb}/_{pulg^{2}} \Big]
\]

Presión atmosférica 
\[
    Pa \;\;
    \Big[ 1 \; atm = 760 \; mmHg = 16.7 {}^{lb}/_{pulg^{2}} = 101.325kPa \Big]
\]

Presión relativa
\[
    P_{rel} 
\]

Presión absoluta
\[
    P_{abs} = P_{atm} \pm P_{rel}
\]

%Figura

\[
    \begin{split}
        & \sum F_{y} = ma_{y} \\
        & P_{y} \Delta z \Delta x - P \Delta s \Delta x \sin{ \theta } = \rho \frac{ \Delta z \Delta y \Delta x }{ 2 } a_{y} \\
        & \sum F_{z} = ma_{z} \\
        & P_{z} \Delta y \Delta x - P \Delta s \Delta x \cos{ \theta } - \rho g \frac{ \Delta z \Delta y \Delta x }{ 2 } = \rho \frac{ \Delta z \Delta y \Delta x }{ 2 } a_{z}
    \end{split}
\]

Pero de la figura se puede ver que
\[
    \begin{split}
        \Delta z = \Delta s \sin{ \theta } \\
        \Delta y = \Delta s \cos{ \theta }
    \end{split}
\]

entonces 
\[
    \begin{split}
        P_{y} - P = \rho \frac{ \Delta y }{ 2 } a_{y} \\
        P_{z} - P = \frac{ \rho }{ 2 } (g + a_{z}) \Delta z
    \end{split}
\]

Si \( \Delta y \to 0 \), \( \Delta z \to 0 \), y \( \frac{ d \Vec{v} }{ dy } = 0 \)
\[
    \begin{split}
        p_{y} - P & = 0 \\
        P_{z} - P & = 0 \\
        P = P_{z} & = P_{y}
    \end{split}
\]

Transmisor de fuerza. Si la presión de un punto es igual a otro
\[
    \begin{split}
        P_{1} & = P_{2} \\
        \frac{ F_{1} }{ A_{1} } & = \frac{ F_{2} }{ A_{2} }
    \end{split}
\]

%figura

Transmisor de presión
\[
    P_{1} A_{1} = P_{2} A_{2}
\]

%figura

si \( \Delta z = 1 \)
\[
    P = P_{x} = P_{y} = P_{z}
\]