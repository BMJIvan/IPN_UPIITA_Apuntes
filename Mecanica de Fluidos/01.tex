\section{Conceptos fundamentales}

\[
    \text{Aplicaciones}
    \left\{
        \begin{array}{l}
            \text{Transporte de fluidos} \\
            \text{Conversión de la energía} \\
            \text{Acondicionamiento de ambiente} \\
            \text{Turbomáquinaria} \\
            \text{Transporte (vehicular)} 
        \end{array}
    \right.
\]

\[
    \begin{array}{c}
         \text{Áreas de}  \\
         \text{mecánica de fluidos}
    \end{array}
    \left\{ 
        \begin{array}{l}
             \text{Aerodinámica} \\
             \text{Hidráulica} \\ 
             \text{Hidrología} \\
             \text{Metrología}
        \end{array}
    \right.    
    \begin{array}{l}
        \\
        \left\{
            \begin{array}{l}
                \text{Hidráulica de potencia} \\
                \text{Redes de tubería} \\
                \text{Neumática}
            \end{array}
        \right. \\
        \\
        \\
    \end{array}
\]

\[
    \begin{array}{c}
         \text{Técnicas o}  \\
         \text{métodos o analíticos}
    \end{array}
    \left\{ 
        \begin{array}{l}
             \text{Analíticos} \\
             \text{Experimentales} \\ 
             \text{Computacionales} 
        \end{array}
    \right.    
    \begin{array}{l}
        \left\{
            \begin{array}{l}
                \text{Diferenciales} \\
                \text{Diferenciales} 
            \end{array}
        \right.\\
        \\
        \\
    \end{array}
\]

Presión, esfuerzo normal: Genera deformaciones lineales
\[
    P = \lim \frac{ \Delta F_{n} }{ \Delta A } = \frac{ dF_{n} }{ dA }
\]

Esfuerzo cortante: Genera deformaciones angulares 
\[
    \tau = \lim \frac{ \Delta F_{t} }{ \Delta A } = \frac{ dF_{t} }{ dA }
\]

\subsection{Propiedades de los fluidos}
Densidad 
\[ 
    \rho = \frac{ m }{ v } \;\; \Big[ {}^{ kg }/_{ m^3 }. {}^{ lbm }/_{ pie^3 }, {}^{ slug }/_{ pie^3 } \Big]
\]

Peso especifico
\[
    \gamma = \frac{ W_{g} }{ v } = \frac{ mg }{ v } = \rho g \;\; \Big[ {}^{ N }/_{ m^3 }, {}^{ lb }/_{ pie^3 } \Big]
\]

Densidad relativa 
\[
    sg = GE = \rho_{r} = \frac{ \rho_{fluido} }{ \rho_{H_{2}O \;\; T = 4^{o}C} }
\]

Viscosidad dinámica o absoluta
\[
    \mu = \frac{ \tau }{{}^{ d \Vec{u} }/{ dy }} \;\; \frac{ \text{Esfuerzo cortante} }{ \text{Gradiente de velocidad} }
\]
\[
    \mu = \frac{ \tau y }{ \Vec{u} } \;\; \Big[ {}^{N \cdot s }/{ m^{2}, {}^{ lb \cdot s }/{ pie^2 } } \Big]
\]

Viscosidad cinemática
\[
    \nu = \frac{ \mu }{ \rho } \;\; \Big[ {}^{ m^{2} }/{ s }, {}^{ pie^{2} }/{ s } \Big]
\]

\subsection{Gases ideales}

Proceso adiabático: Aquel proceso en el que no se gana ni pierde calor, es decir, cuenta con un aislamiento térmico. 

En proceso adiabático reversible no hay transferencia de calor y por lo tanto el proceso es isoentrópico.

Un proceso adiabático irreversible no es isoentropico. 
\[
    \begin{split}
        \forall & : \text{ Volumen} \\
        \nu & : \text{ volumen especifico \( \frac{ \forall }{ m } = \frac{ 1 }{ \rho } \)}
    \end{split}
\]

\begin{enumerate}
    \item Ley de Boyle y Mariotte
        \[
            \begin{split}
                \text{Si }T & = constante \\
                P \; & \alpha \; \frac{ 1 }{ \forall } \\
                P \forall & = C \\
                P_{ 1 } \forall_{ 1 } & = P_{ 2 } \forall_{ 2 }  
            \end{split}
        \]
    \item Ley de Charles
        \[
            \begin{split}
                \text{Si }P & = constante \\
                \forall\; & \alpha \; T \\
                \frac{ \forall }{ T } & = C \\
                \frac{ \forall_{1} }{ T_{1} } & = \frac{ \forall_{2} }{ T_{2} } 
            \end{split}
        \]
    \item Ley de Gay - Lussac
        \[
            \begin{split}
                \text{Si } \forall & = constante \\
                P \; & \alpha \; T \\
                \frac{ P }{ T } & = C \\
                \frac{ P_{1} }{ T_{1} } & = \frac{ P_{2} }{ T_{2} } 
            \end{split}
        \]
    \item 
        \[
            \begin{split}
                \frac{ P \forall }{ T } & = C \\
                \frac{ P_{1} \forall_{1} }{ T_{1} } & = \frac{ P_{2} \forall_{2} }{ T_{2} } \\
                \frac{ P \nu }{ T } & = nR_{u} \\
                R_{u} & = \frac{ P \forall }{ Tn }
            \end{split}
        \]
        Donde 
        \[
            \begin{split}
                P & = 1 \text{ atmósfera} \\
                T & = 0^{o}C = 273.15K \\
                n & = 1 \text{ kmol} \\
                \forall & = 22.413 m^{3} \\
                R_{u} & = 8.314 {}^{kJ}/_{kmol \cdot K} \\
                n & = \frac{ m }{ M } \; \frac{ \text{ masa } }{ \text{ masa molar } } \\
                R & = \frac{ R_{u} }{ M } \text{ constante del gas }
            \end{split}
        \]
        Entonces 
        \[
            \begin{split}
                R_{u} & = \frac{ P \forall }{ T {}^{m}/{M} } = \frac{ MP \forall }{ Tm } \\
                m \frac{ R_{u} }{ M } & = \frac{ P \forall }{ T } \\
                \frac{ P \forall }{ T } & = mR \\
                P \forall & = TmR \\
                P \frac{ \forall }{ m } & = TR \\
                P \nu & = TR \\
                P & = \frac{ 1 }{ \nu } TR = \rho TR \\
                \frac{ P }{ \rho } & = TR
            \end{split}
        \]
\end{enumerate}

\subsection{ Velocidad sonica o acústica y viscosidad }

A partir de 1
\[
    dp \; \alpha \; - \frac{ d \forall }{ \forall } \\
    dp = -E_{v} \frac{ d \forall }{ \forall }
\]

Donde \( E_{v} \) es el modulo de compresibilidad o modulo volumétrico. \\ 
De la definición de masa
\[
    \begin{split}
        m & = \rho \forall \\
        dm & = \rho dv + d\rho \forall, \;\; dm = 0 \\
        -\rho d\forall & = d\rho \forall \\
        - \frac{ d \forall }{ \forall } & = \frac{ d\rho }{ \rho }
    \end{split}
\]

por lo tanto
\[
    dp = E_{v} \frac{ d\rho }{ \rho }
\]

\[
    C = \sqrt{ \frac{ dp }{ d\rho } } = \sqrt{ \frac{ E_{v} }{ \rho } } \;\; \text{Velocidad del sonido a través de líquidos}
\]
\[
    \begin{split}
        K & = \frac{ C_{p} }{ C_{\nu} } = 1.4 \\
        R & = C_{p} - C_{ \nu } \;\; \Big[ {}^{kJ}/{ kg \cdot K } \Big] \\
        R & = .287 {}^{kJ}/{ kg \cdot K } \\
        C & = \sqrt{ \frac{ k_{p} }{ \rho } } = \sqrt{KTR} \\
        E_{v} & = P \; \text{ Proceso isotérmico} \\
        E_{v} & = KP \; \text{ Proceso isoentropico} \\
    \end{split}
\]
Líquidos incompresibles \( \rho = constante \) \\
Líquidos compresibles \( \rho \not = constante \) 
\[
    \begin{split}
        \text{Mach } & = \frac{ \Vec{v} }{ C } \; \frac{ \text{velocidad fluido} }{ \text{ velocidad de sonido } } \\
        \text{Mach } & \leq .3 \; \text{flujo de gas incompresible} \\
        \text{Mach } & \geq .3 \; \text{flujo compresible}
    \end{split}
\]

\[
    \overbrace{ \frac{ \mu }{ \rho } }^{ \text{Dinámica} } = \overbrace{ v }^{ \text{Cinemática} }
\]

\[
    \begin{split}
        \mu & = \text{ constante o \( 0 \) ideal o no viscoso} \\
        \mu & \not = \text{ constante o \( 0 \) real o viscoso }
    \end{split}
\]

\[
    \begin{array}{c}
         \text{Fluido de acuerdo}  \\
         \text{al comportamiento} \\
         \text{de la \( \mu \)}
    \end{array}
    \left\{ 
        \begin{array}{l}
             \text{Newtoniano} 
                \left\{
                    \begin{array}{l}
                        \tau \; \alpha \; \frac{ d\theta \text{ deformación} }{ dt } \to 
                            \begin{array}{l}
                                \text{ley de viscosidad} \\
                                \text{de Newton}
                            \end{array}
                    \end{array}
                 \frac{ d\theta }{ dt } = \frac{ d\Vec{v} }{ dy } 
                 \right.\\ 
             \text{No Newtoniano} 
                \left\{
                    \begin{array}{l}
                         \tau \text{ no } \; \alpha \; \frac{ d\theta }{ dt } \to \text{ Series de potencias }
                    \end{array}
                \right.\\ 
             \text{Visco-elástico} 
                \left\{
                    \begin{array}{l}
                         \text{Comportamiento Newtoniano}  \\
                         \text{y no Newtoniano}
                    \end{array}
                \right.
        \end{array}
    \right.    
\]

Número de Raynolds
\[
    NR_{E} = \frac{ \text{Fuerzas de inercia} }{ \text{Fuerzas viscosas} } = \overbrace{ \frac{ \rho \Vec{v} D }{ \mu } }^{ \text{Dinámica} } = \overbrace{ \frac{ \Vec{v}D }{ \forall } }^{ \text{Cinemática} }
\]

\[
    \begin{array}{c}
         \text{Flujo viscoso}
    \end{array}
    \left\{ 
        \begin{array}{l}
             \text{flujo laminar} 
                \left\{
                    \begin{array}{l}
                        NR_{E} \leq 2000 \;\; (2300) 
                    \end{array}
                 \right.\\ 
             \text{flujo transición} 
                \left\{
                    \begin{array}{l}
                         2000 \leq NR_{E} \leq 4000
                            \begin{array}{l}
                                \text{puede ser}\\
                                \text{laminar o turbulento}
                            \end{array}
                    \end{array}
                \right.\\ 
             \text{flujo turbulento} 
                \left\{
                    \begin{array}{l}
                         NR_{E} \geq 4000
                    \end{array}
                \right.
        \end{array}
    \right.    
\]

\[
    P_{abs} = P_{atm} \pm P_{rel} \to P_{ \text{manométrica}}
\]

\[
    \begin{split}
        Mach & > 1 \; \text{ supersónico} \\
        Mach & > 5 \; \text{ hipersónico} \\
        Mach & < 1 \; \text{ sursónico} \\
        Mach & = 1 \; \text{ sónico} \\
        Mach & \leq 1 \; \text{ transónico}
    \end{split}
\]

\[
    \dot{ \forall } = \frac{ \forall }{ t } = \Vec{ v } A \to \; \text{ caudal}
\]

\[
    \begin{split}
        \rho_{ gasolina } = 680 {\;}^{ kg }/_{ m^{3} } \;\; & E_{v} = 1.3 \times 10^{9} {\;}^{ N }/_{ m^{2} } \\
        \rho_{ Hg } = 13600 {\;}^{ kg }/_{ m^{3} } \;\;& E_{v} = 2.85 \times 10^{10} {\;}^{ N }/_{ m^{2} } \\
        \rho_{H_{2}O \; mar} = 1030 {\;}^{ kg }/_{ m^{3} } \;\;& E_{v} = 2.34 \times 10^{9} {\;}^{ N }/_{ m^{2} } \\
    \end{split}
\]

\subsection{Esfuerzo cortante}
\[
    \begin{split}
        \tau \; & \alpha \; \frac{ du }{ dt } = \frac{ \Vec{v} }{ dy } \\
        \tau & = \mu \frac{ d\Vec{ v } }{ dy } \\
        \tau \int_{ 0 }^{ h }dy & = \mu \int_{ 0 }^{ \Vec{v} } \\
        \tau h & = \mu \Vec{ v } \\
        \tau & = \mu \frac{ \Vec{ v } }{ h } \\
        \tau & = \lim_{\Delta A \to 0} \frac{ \Delta F_{t} }{ \Delta A } = \frac{ dF_{t} }{ dA } \\
        \tau & = \frac{ F_{t} }{ A }
    \end{split}
\]
Donde \( F_{v} \) es la fuerza viscosa o tangente
\[
    \begin{split}
        \mu \frac{ \Vec{v} }{ h } = \frac{ F_{v} }{ A }\\
        F_{v} = \frac{ \mu \Vec{v} A }{ h }    
    \end{split}
\]

caso 1: 
\[
    \begin{split}
        A & = \pi d L \\
        h & = \frac{ D - d }{ 2 } \\
        F_{v} & = \frac{ 2 \pi d L \Vec{v} \mu }{ D - d }
    \end{split}
\]

caso 2:
\[
    \begin{split}
        F_{v} & = W \sin{ \theta } \\
        h & = \frac{ \mu \Vec{v} A }{ W \sin{ \theta } }
    \end{split}
\]

*tablas líquidos y gases: mecánica de fluidos Pottev

\subsection{Tensión o esfuerzo superficial} 
\[
    \sigma_{s} = \frac{ F_{ \text{tensión} } }{ l } \;\; \Big[ {}^{ W }/_{ m }, {}^{ lb }/_{ pie } \Big]
\]
\[ 
    \begin{split}
        \sum F_{y} & = 0 \\
        F_{rsy} - Wg & = 0 \\
        \sigma_{s} \pi D \cos{ \theta } - \rho \forall g & = 0 \\
        \sigma_{s} \pi D \cos{ \theta } - \rho g \frac{ \pi D^{2} }{ 4 } h & = 0 \\
        \sigma_{s} \cos{ \theta } & = \frac{ \rho g D h }{ 4 } \\
        h & = \frac{ 4 \sigma_{s} \cos{ \theta } }{ \rho g D }
    \end{split}
\]