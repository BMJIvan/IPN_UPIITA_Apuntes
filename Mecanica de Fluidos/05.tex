\section{Flujo incompresible en tuberías}
\[
    \begin{split}
        \frac{ \dot{Q} }{ \dot{m} } & = q = \frac{ Q }{ \Delta t } \\
        \frac{ \dot{W} }{ \dot{m} } & = w = \frac{ W }{ \Delta t }
    \end{split}
\]
\[
    \begin{split}
        \dot{Q}_{ent} + \dot{W}_{ent} & = \dot{m} ( \Delta P_{v} + \Delta u + \Delta e.c + \Delta e.p) + \dot{Q}_{sal} + \dot{W}_{sal} \\
        q_{ent} + w_{ent} & = (P_{2}v_{2} - P_{1}v_{1}) + (\mu_{2} - \mu_{1}) + \frac{ v_{2}^{2} - v_{1}^{2} }{ 2 } + g(z_{2} - z_{1}) + q_{sal} + w_{sal} \\
        \frac{ P_{1} }{ \rho g } + \frac{ v_{1}^{2} }{ 2g } + z_{1} & = \frac{ P_{2} }{ \rho g } + \frac{ v_{2}^{2} }{ 2g } + \frac{ (u_{2} - u_{1}) }{ g } + \frac{ (q_{sal} - q_{ent}) }{ g } + \frac{ w_{sal} - w_{ent} }{ g }
    \end{split}
\]

Perdidas totales del sistema
\[
    hlf = \Big[ \frac{ u_{2} - u_{1} }{ g } + \frac{ q_{sal} - q_{ent} }{ g }\Big] = h_{f} + h_{m}
\]

\( hlfs \): perdidas totales en la succión 

Turbina, motores
\[
    h_{T} = h_{m} = \frac{ w_{sal} }{ g }
\]

Bomba, ventilador
\[
    \begin{split}
        h_{B} & = h_{v} = \frac{ w_{ent} }{ g } \\
        & = z_{2} + hlf_{1 \to 2} \\
        & = \underbrace{ h_{suc} + h_{des} }_{ z_{2} } + hlfs + hlfd
    \end{split}
\]

Potencia que proporciona una bomba al agua
\[
    \begin{split}
        h_{B} & = \frac{ w_{b} }{ g } = \frac{ \dot{w_{b}} }{ \dot{m} } \\
        \dot{w}_{B} & = PH = h_{B} g \dot{m} \\
        PH & = h_{B} \cdot g \cdot \rho \cdot \dot{ \forall } = \gamma h_{B} \dot{ \forall }
    \end{split}
\]

Parámetros para elegir una bomba, compresor. Para un modelo x
\[
    \underbrace{ 6" }_{
        \begin{array}{cc}
            \text{ diámetro }\\
            \text{ succión }
        \end{array}} \times
    \underbrace{ 5" }_{
        \begin{array}{cc}
            \text{ diámetro }\\
            \text{ descarga }
        \end{array}} \times
    \underbrace{ 10" }_{
        \begin{array}{cc}
            \text{ diámetro }\\
            \text{ máximo }
        \end{array}}
\]

Carga neta positiva de succión
\[
    \begin{split}
        NPSH & = 
        \begin{array}{cc}
            \text{ carga de presión } \\
            \text{ de estancamiento } \\
            \text{ absoluta } 
        \end{array} -
        \begin{array}{cc}
             \text{ carga de presión }\\
             \text{ de vapor }
        \end{array} \\
        NPSH & = \Big[ \frac{ P_{atm} }{ \rho g } + \frac{ P_{s} }{ \rho g } + \frac{ V_{s}^{2} }{ 2g } \Big]_{ \text{estática} } - \frac{ Pv }{ \rho_{s} } \\
        NPSH_{disponible} & = \underbrace{ \frac{ P_{atm} }{ \rho g } }_{
            \begin{array}{cc}
                \text{ presión }\\
                \text{ al bombeo }
            \end{array}                                 
        } \pm h_{s} - hlfs - \frac{ P_{vapor} }{ \rho g }: \;\; \text{ \( P_{vapor} \) de tablas }
    \end{split}
\]

\subsection{Caídas de presión en tuberías y accesorios}
%figura
Tuberías
\[
    \begin{split}
        \frac{ P_{1} - P_{2} }{ \rho g } & = hlf_{1 \to 2} = \underbrace{ h_{f} }_{
            \begin{array}{cc}
                \text{ perdidas por } \\
                \text{ fricción en } \\
                \text{ tuberías }
            \end{array}             
        } + \underbrace{ h_{m} }_{ 
            \begin{array}{cc}
                \text{ perdidas por } \\
                \text{ accesorios }
            \end{array}
        } \;\; hm \to 0 \\
        \Delta P & = P_{1} - P_{2} = \rho g h_{f} = \gamma h_{f}
    \end{split}
\]

Bridas, unión
\[
    \begin{split}
        \frac{ P_{1} }{ \rho g } + \frac{ \Vec{v}_{1}^{2} }{ 2g } + z_{1} & = \frac{ P_{2} }{ \rho g } + \frac{ \Vec{v}_{2}^{2} }{ 2g } + z_{2} + hlf_{1 \to 2} + h_{B} + h_{T} \\
        \Delta P & = \rho g h_{m} = \gamma h_{m}
    \end{split}
\]

\[
    \begin{split}
        h_{m} \; & \alpha \; \frac{ \Vec{v}^{2} }{ 2g } \\
        h_{m} & = k \frac{ \Vec{v}^{2} }{ 2g }
    \end{split}
\]

factor de fricción del accesorio
\[
    k = f_{acc} \times \frac{ L_{equivalente} }{ \text{ Diámetro } }
\]

Ecuación de Darcy
\[
    hf = f \frac{ L }{ D } \frac{ \Vec{v}^{2} }{ 2g } \\
\]
donde \( f \) es el factor de fricción de tubería 

Ecuación de poiseuille: para flujo laminar
\[
    \begin{split}
        f & = \theta(NRe) \\
        & = \frac{ 64 }{ NRe }
    \end{split}
\]

Correlación de Blasius 

Valida para flujo turbulento en tuberías lisas y \( NRe <  10^{5} \) \\
\[
    \begin{split}
        f & = \theta(NRe, \frac{ D }{ e }): \;\; e: \text{rugosidad relativa} \\
        & = \frac{ .3164 }{ NRe^{.25} }
    \end{split}
\]

Correlación de Shamee-Jain
\[
    f = \frac{ .25 }{ \Big[ log \Big( \frac{ 1 }{ 3.1 ( {}^{D}/_{e} ) } + \frac{ 5.74 }{ NRe^{.9} } \Big) \Big]^{ 2 } } 
\]