\section{Definiciones}
\begin{table}[h]
    \centering
    \begin{tabular}{|c|l|l|} \hline
        \thead{Mecatronics\\journal} & \thead{1991\\M, E}& Balance óptimo de estructuras mecánicas y control \\ \hline
        IEEE/ASME & \thead{1996\\M, E, C} & \thead{Integración sinérgica, diseño y manufactura de producto\\ y procesos}\\ \hline
        Bishop & \thead{2002\\M, E, I} & Proceso en evolución, integración, sinérgica \\ \hline
        Amerogen & \thead{2003\\M, E} & Diseño como un todo \\ \hline
        Iserman & \thead{2005\\M, E, I, C} & \thead{Integrar sistemas, diseño simultaneo, balance óptimo entre\\ estructuras, actuadores, sensores, información y control} \\ \hline
        Bolton & \thead{2008\\M, E, C} & \thead{Completa integración } \\ \hline
        De Silva & \thead{2008\\M, E, I, C} & \thead{Sinergia y diseño integrado, sistemas electromecánicos, grado de\\ inteligencia, más preciso, exacto, seguro, flexible, funcional y\\ mecánicamente menos complejo} \\ \hline
        Sheltty y Kolk & \thead{2011\\M, E, I} & \thead{Diseño óptimo, productos electromecánicos, concurrencia y\\ sinergia } \\ \hline
        Merzouki & \thead{2013\\M, E, C} & \thead{Mejorar sistemas mecánicos con control inteligente. Remplazar\\ componentes mecánicos con electrónica} \\ \hline
        Cetinkowt & \thead{2015\\M, E, I} & \thead{Integración sinérgica, nivel de sistemas. Diseño simultaneo\\ para obtener un diseño óptimo } \\ \hline
    \end{tabular}
    \caption{Definiciones}
    \label{tab:Definiciones}
\end{table} 