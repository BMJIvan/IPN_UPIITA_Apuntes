\section{Selección de diseño conceptual y componentes}

\subsection{Criterios de selección para el diseño conceptual}

Un criterio de selección debe permitir elegir y comparar entre las opciones de diseño conceptual generados. Esta selección debe ser clara, no debe dejar a dudas para el diseñador. 

Los criterios deben proponerse con base en las funciones y comportamientos que el sistema debe realizar, es decir, se debe buscar definir criterios que permitan "medir" el rendimiento de las funciones. 

Deben estar fundamentadas en las necesidades y requerimientos definidos anteriormente. 

Se recomienda definir al menos tres criterios por cada característica que conforma el diseño conceptual. 

\subsection{Selección de componentes}

Un componente se define como un elemento que desempeña una función o varias funciones de un sistema. Estos componentes son los que el diseñador no va a diseñar, sino seleccionar. Los componentes son también denominados partes. 

Para la selección de componentes se emplearán dos herramientas.

\begin{enumerate}
    \item Árbol de decisión: permite realizar una ruta y clasificación de los componentes de forma general.
    
    \item Método de selección multicriterio: Permitiendo definir el componente "único" que se empleará a través de los criterios de selección.
\end{enumerate}

Para la selección de los componentes se necesitan los requerimientos y funciones que va a desempeñar.

\subsection{Proceso de selección de componentes}
\begin{enumerate}
    \item Identificar el tipo de componente que se requiere con base en las funciones que debe desempeñar, por ejemplo, medición de temperatura, generación de movimiento lineal. 
    
    \item Con base en los requerimientos, se requiere una investigación profunda para determinar cuáles pueden ser los fabricantes y proveedores. Estos siempre serán los que puedan aportar mayor información.
    
    \item Una vez identificados los proveedores (fabricantes), investigar y analizar los catálogos de sus productos. Es aconsejable que se revisen al menos tres fabricantes (si es que existen). 
    
    Es recomendable que se contacte con el fabricante para explicar las funcionalidades y requerimientos del componente, esto facilita la selección. Puede que recomienden un modelo o una familia.
    
    \item Aplicar el árbol de decisión a través de la información analizada de los catálogos, funcionalidades y requerimientos. Se pueden hacer preguntas que determinen y acaten el universo de componentes posible. 
\end{enumerate}