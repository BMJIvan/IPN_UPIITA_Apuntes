\section{Diseño conceptual}
Con base en los sistemas que conforman la arquitectura física se requiere comenzar a buscar posibles soluciones. Estas soluciones deben cumplir las funciones, los requerimientos y las necesidades.

La integración de las soluciones por cada uno de los módulos se denomina diseño conceptual.

\begin{enumerate}
    \item Análisis morfológico, caja morfológica
        \begin{itemize}
            \item De cada característica se requiere al menos una posible solución. 
            \item Posibles soluciones, no importa si parecen "absurdas".
            \item Mayor cantidad de posibles soluciones.
            
        \end{itemize}
        
        No
        \begin{itemize}
            \item Componentes
            \item Materiales
            \item Procesos de manufactura
        \end{itemize}
        
        Si
        \begin{itemize}
            \item Mecanismos 
            \item Controladores
            \item Formas
            \item Métodos
            \item Ubicación componentes 
        \end{itemize}
        
        \item Generar combinaciones
        \begin{itemize}
            \item Deben ser congruentes
            \item De 3 a 5 combinaciones
            \item Deben incluir todas las características
            \item Deben buscar el mejor desempeño de las funciones
        \end{itemize}
        Cada combinación se denomina "Diseño conceptual"
        
        \item Seleccionar el mejor diseño conceptual
        \begin{itemize}
            \item Generar los criterios de selección
            \item Determinar una herramienta de selección multicriterio 
            \item Resolver el problema de selección
        \end{itemize}
        
        \item Analizar resultados del método de selección
        
        \item Comprobar
\end{enumerate}