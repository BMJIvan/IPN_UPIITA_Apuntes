\section{Validación y verificación}

El proceso de validación y verificación es transversal o paralelo ya que se desarrolla a lo largo del proceso de diseño. Usualmente se desarrolla con ayuda de los procesos de modelado y simulación.

\subsection{Validación}

Validación es el proceso para determinar si se ha diseñado el sistema en forma correcta, y cumple con índices de desempeño, y las necesidades-requerimientos.

Algunas de las validaciones existentes son 
\begin{itemize}
    \item Validación operacional
    \item Validación concepto
    \item Validación requerimiento
    \item Validación función 
\end{itemize}

\subsection{Verificación}

La verificación es la coincidencia de los componentes, ensambles, módulos y sistemas en los requerimientos para asegurar que se han logrado correctamente. 

Dentro de las técnicas y herramientas que existen para verificar el sistema, se encuentran las siguientes

\begin{itemize}
    \item Análisis del sistema
    \item Estructura del sistema
    \item Análisis del valor (ingeniería del valor)
    \item Análisis de requerimientos (QFD)
\end{itemize}

\subsection{Validación y verificación del sistema}

Existen 4 métodos para la evaluación completa de un sistema
\begin{itemize}
    \item Pruebas instrumentales con equipo calibrado
    \item Análisis y simulaciones, computacionales y analíticas
    \item Demostraciones o pruebas funcionales ante un jurado
    \item Examinación de la documentación 
\end{itemize}

El desarrollo de prototipos a lo largo del proceso de diseño, permite que los interesados del proyecto validen y verifiquen de forma constante.

El proceso termina con la aceptación del sistema.

Un prototipo es un modelo físico del sistema que no incluye todos los aspectos, pero permite analizar algunas partes o funcionalidades específicas.  

La validación de los procesos de manufactura y ensamble son requeridos, el análisis de la secuencia del ensamble, las tolerancias de manufactura, minimizar costos de manufactura, ensamble, herramientas, tiempos, entre otras cosas.

\subsection{Matriz de trazabilidad}

Para determinar el cumplimiento de las necesidades, requerimientos, funciones y sistemas, se emplean a lo largo del proceso de diseño, las matrices de trazabilidad. 
\begin{enumerate}
    \item Necesidades-requerimientos
    \item Requerimientos-funciones
    \item funciones-sistemas/modelos
\end{enumerate}

Todas las necesidades deben tener al menos un requerimiento asociado.

Una función sin requerimientos: hace más de lo que debería hacer.

Si no existen funciones asociadas a todos lo requerimientos, quizá falta definir funciones. 

Índice: Valor tangible para evaluar