\section{Comportamiento entre funciones}
Para describir el comportamiento se tiene cuatro tipos de operadores lógicos.

\begin{enumerate}
    \item AND
    \item OR (paralelo), XOR (no se puede en paralelo)
    \item LOOP
    \item ITERACION
\end{enumerate}

Estos operadores se representan con un círculo y su símbolo se emplean deben aparecer al inicio y al fin de su uso. En el caso de los operadores "it" y "LP", se deben incluir la condición a la que están sujetas. 

Al inicio y al término del modelo se deben incluir funciones o acciones de referencias: se expresan como:

Existen otro tipo de funciones que son tentativas, es decir, no siempre existen. Estas se representan con bloques punteados. 

Si existieran elementos actuadores del comportamiento, estos se modelan con rectángulos con aristas redondeadas.

Para evaluación del diseño se emplea el coeficiente de diseño mecatrónico (MQD). Con los siguientes atributos.
\begin{enumerate}
    \item Inteligencia
    \item Robusto
    \item Flexible
    \item Adaptable
    \item Eficiencia del ensamble
\end{enumerate}