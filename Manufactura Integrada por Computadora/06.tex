\section{CAQ}

Calidad 
\begin{itemize}
    \item Expectativa \( \to \) necesidades
    
    \item Requerimiento \( \to \) ingeniería 
        \begin{itemize}
            \item Medibles 
            \item Alcanzables
            \item Especifico 
            \item Verificación \( \to \) inspección
        \end{itemize}
    \item Funcionamiento (desempeño)
        \begin{itemize}
            \item Indicadores dde desempeño KPI (Key Performance Indicator), instrumento de medición
        \end{itemize}
\end{itemize} 

Los KPIs son métricas que nos ayudan a identificar el rendimiento de una determinada acción o estrategia. Estas unidades de medida nos indican nuestro nivel de desempeño con base en los objetivos que se han fijado con anterioridad. 

Es necesario comparar periódicamente los resultados que estamos obteniendo con los objetivos fijados. Esto nos permitirá averiguar si vamos por un buen camino o si existen desviaciones negativas. 

Si no estamos obteniendo los resultados esperados, los KPIs permitirán darnos cuenta y poder reaccionar a tiempo.

Los KPIs se agrupan en cuadros de mando para que los directivos puedan ser agiles en la toma de decisiones. En el cuadro de mando se incluyen los principales indicadores clave para la empresa, y de una forma visual se obtiene la información deseada de nuestro rumbo sobre el plan establecido. 

\begin{itemize}
    \item Medibles: Son medibles en unidades
    \item Alcanzables: Si se puede medir, se puede cuantificar
    \item Especifico: Se debe centrar en un único aspecto a medir, hay que ser concretos
    \item Temporal: Debe poder medirse en el tiempo (diario, semanal, mensual)
    \item Relevante: Únicamente sirven aquellos factores que sean relevantes para la empresa
\end{itemize}

Los KPIs tienen que informar, controlar, evaluar y por último ayudar a que se tomen decisiones. Cada empresa tiene sus propios indicadores de gestión, puesto que cada organización y cada modelo de negocio tienen factores a medir diferentes. Una empresa de producción industrial hará foco en indicadores de producción y una empresa que únicamente venda a través de internet tendrá otros indicadores clave relacionados con métricas de marketing digital. 

\subsection{Atributos de sistema mecatrónico}
\begin{itemize}
    \item Adaptable
    \item Robusto
    \item Flexible 
    \item Inteligente 
    \item Modular: disminuir, reducir la dependencia. 
\end{itemize}