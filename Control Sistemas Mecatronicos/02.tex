\section{Transformación de similitud}
considere el vector \(q:n\times1\;(q\in R^n)\). El conjunto de vectores \(q_{1}, \ldots, q_{m}\) es linealmente independiente si existen numero reales \(\alpha_{1}, \ldots, \alpha_{m}\) no todos cero, tales que
\[
    \alpha_{1}q_{1} + \alpha_{1}q_{1} \ldots, \alpha_{n}q_{n}=0 \;\; (1)
\]

Si la solución unica de (1) es \(\alpha_{1}=\alpha_{2} \ldots =\alpha_{m}\) entonces el conjunto de vectores es linealmente independientes (l.i).

A la expresión \(\alpha_{1}q_{1}+\alpha_{2}q_{2}+ \ldots +\alpha_{n}q_{n}\) se le denomina combinación lineal.

Base: Un conjunto de vctores l.i en \(R^n\) se define como una base si se puede expresar como una combinación lineal unica.

En \(R^n\) todo conjunto de vectores l.i puede utilizarse como una base.

Sea \(X:n\times1\) todo vector \(X\) puede expresarse como 
\[
    X=\alpha_{1}q_{1}+ \ldots\ +\alpha_{n}q_{n}\;\; (2)
\] 
donde \(q_{i}\) son l.i.

De (2) se tiene que
\[
    X=
    \underbrace{
        \begin{bmatrix}
            q_{1},& q_{2},& \ldots,& q_{n}
        \end{bmatrix}
                }_{n\times n}
    \underbrace{
        \begin{bmatrix}
            \alpha_{1}\\
            \alpha_{2}\\
            \vdots\\
            \alpha_{n}
        \end{bmatrix}
                }_{n\times 1}\;\;(3)
\]

se definen
\[
    Q=
    \begin{bmatrix}
        q_{1},& q_{2},& \ldots,& q_{n}
    \end{bmatrix},\;\;
    \tilde{X}=
    \begin{bmatrix}
        \alpha_{1}\\
        \alpha_{2}\\
        \vdots\\
        \alpha_{n}
    \end{bmatrix}
\]

entonces sustituyendo en (3) se tiene que
\[
    X=Q\tilde{X}
\]

donde \(X\) y \(\tilde{X}\) son similares.

De (2) \(X^\top = \alpha_{1}S_{1} + \alpha_{2}S_{2} + \ldots + \alpha_{n}S_{n}\;\; S_{i}:1\times n\)

\[
    X^\top = 
        \begin{bmatrix}
            \alpha_{1}, & \alpha_{2}, & \ldots & \alpha_{n}
        \end{bmatrix}
        \begin{bmatrix}
            S_{1}\\
            S_{2}\\
            \vdots\\
            S_{n}
        \end{bmatrix}
\]

Una matriz es estable cuando los valores propios son negativos.

Sea la ecuacion lineal 
\[
    Ax=y\;\;(4)
\]

donde \(A:n\times n\;\; B:n\times 1\;\; y:n\times 1\)
se definen
\[
    x=Q\tilde{x}, \;\; y=Q\tilde{y}
\]

sustituyendo en (4) se tiene que 
\[
    \begin{split}
        AQ\tilde{x} & = Q\tilde{y}\\
        Q^{-1}AQ\tilde{x} & =\tilde{y} \;\; (5)
    \end{split}
\]

donde \(A\) y \(Q^{-1}AQ\) son similares y \(A\) esta relacionada con la estabilidad.

ejercicio: Sea \(A:n\times n\) una matriz estable, demuestre que la matriz \(\tilde{A} = Q^{-1}AQ\) es tambien estable, considere que \(Q\) es invertible.

Si
\[
    \begin{split}
        det(\lambda I-A) & = \lambda^n + alpha_{1}\lambda^{n-1} + \alpha_{2}\lambda^{n-2} + \ldots + \alpha_{n} = 0\\
        det(\lambda I-\tilde{A}) & = \lambda^n + alpha_{1}\lambda^{n-1} + \alpha_{2}\lambda^{n-2} + \ldots + \alpha_{n} = 0
    \end{split}
\]
entonces
\[
    \begin{split}
        det(\lambda I-\tilde{A} & = det(\lambda I-A )\\
        det(\lambda I-\tilde{A}) & = det(\lambda Q^{-1}Q-Q^{-1}AQ)\\
        & = det(Q^{-1}\lambda Q-Q^{-1}AQ)\\
        & = det(Q^{-1}(\lambda I-A)Q)\\
        & = det(Q^{-1}) det(\lambda I-A) det(Q)\\
        & = det(Q^{-1}) det(Q) det(\lambda I-A)\\
        & = det(Q^{-1}Q) det(\lambda I-A)\\
        & = det(\lambda I-A)
    \end{split}
\]

repetir el ejercicio anterior considerando la siguiente matriz \(\tilde{A}=QAQ^{-1}\)

\[
    \begin{split}
        det(\lambda I-\tilde{A} & = det(\lambda I-A )\\
        det(\lambda I-A) & = det(\lambda I-QAQ^{-1})\\
        & = det(\lambda QQ^{-1}-QAQ^{-1})\\
        & = det(Q\lambda Q^{-1}-QAQ^{-1})\\
        & = det(Q(\lambda I-A)Q^{-1})\\
        & = det(Q) det(\lambda I-A) det(Q^{-1})\\
        & = det(Q) det(Q^{-1}) det(\lambda I-A)\\
        & = det(QQ^{-1}) det(\lambda I-A)\\
        & = det(\lambda I-A)
    \end{split}
\]

