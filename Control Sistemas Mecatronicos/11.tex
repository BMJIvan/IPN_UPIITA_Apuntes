\section{Formula de Ackerman}

El metodo consiste en calcular la ganacia \( k \) para asignar los polos en lazo cerrado \( \mu_{1}, \mu_{2}, \ldots, \mu_{n} \) a partir del polinomio caracteristico en lazo cerrado
\[
    P_{LC}(s) = s^{n} + \tilde{a}_{1}s^{n-1} + \tilde{a}_{2}s^{n-2} + \ldots + \tilde{a}_{n} = 0 
\]

y en lazo abierto se tiene
\[
    P(s) = s^{n} + a_{1}s^{n-1} + a_{2}s^{n-2} + \ldots + a_{n} = 0
\]

Teorema de Cayley-Hamilton \\
Toda matriz satisface su polinomio caracteristico, por lo tanto se tiene que
\[
    \begin{split}
        P(A) & = A^{n} + a_{1}A^{n-1} + a_{2}A^{n-1} + \ldots + a_{n}I = 0\\
        P_{LC}(A-Bk) & = (A-Bk)^{n} + \tilde{a}_{1}(A-Bk)^{n-1} + \tilde{a}_{2}(A-Bk)^{n-2} + \ldots + \tilde{a}_{n}I = 0
    \end{split}
\]

Para \( n=3 \) se tiene que 
\[
    \begin{split}
        (A-Bk)^{2} & = A^{2} - ABk - Bk(A-Bk)\\
        (A-Bk)^{3} & = A^{3} - A^{2}Bk - ABk(A-Bk) -Bk(A-Bk)^{2}
    \end{split}
\]

Por lo tanto el polinomio caracteristico en lazo cerrado se puede escribir de la siguiente forma
\[
    \begin{split}
            P_{LC}(A-Bk) & = A^{3} - A^{2}Bk - ABk(A-Bk) -Bk(A-Bk)^{2} \\ & \;\;\;\; + \tilde{a}_{1}(A^{2} - ABk - Bk(A-Bk)) + \tilde{a}_{2}(A-Bk) + \tilde{a}_{3}I \\
            & = A^{3} + \tilde{a}_{1}A^{2} + \tilde{a}_{2}A + \tilde{a}_{3}I - A^{2}B(k) - AB(k(A-Bk)) \\
            & \;\;\;\; - B(K(A-Bk)^{2}) - AB(\tilde{a}_{1}k) - B(\tilde{a}_{1}k(A-Bk)) -B(\tilde{a}_{2}k)\\
            & = \underbrace{A^{3} + \tilde{a}_{1}A^{2} + \tilde{a}_{2}A + \tilde{a}_{3}I}_{P_{LC}(A)} -B(k(A-Bk)^{2}+\tilde{a}_{1}k(A-Bk)-\tilde{a}_{2}k) \\ & \;\;\;\; -AB(k(a-Bk)+\tilde{a}_{1}k) - A^{2}B(k) \\
            & = P_{LC}(A) - 
            \underbrace{
                \begin{bmatrix}
                    B & BA & A^{2}B
                \end{bmatrix}}_
                {
                \begin{scriptsize}
                    \begin{matrix}
                        \text{Matriz de}\\
                        \text{controlabilidad}\;C
                    \end{matrix}
                \end{scriptsize}
                }
            \underbrace
            {
            \begin{bmatrix}
                k(A-Bk)^{2} & \tilde{a}_{1}k(A-Bk) & \tilde{a}_{2}k \\
                0 & k(A-Bk) & \tilde{a}_{1}k\\
                0 & 0 & k
            \end{bmatrix}
            }_{G}\\
            & = P_{LC}(A) -CG = 0
    \end{split}
\]

Se despeja la ganancia \( G \)
\[
    \begin{split}
        P_{LC}(A) - CG & = 0 \\
        -CG & = P_{LC}(A) \\
        CG & = P_{LC}(A) \\
        G & = C^{-1}P_{LC}(A)
    \end{split}
\]

Se tiene un sistema de ecuaciones, pero solo se va a tomar el ultimo elemento de la matriz de ganancias, por lo tanto
\[
    \begin{split}
        \begin{bmatrix}
            0 & 0 & 1    
        \end{bmatrix}
        G & = 
        \begin{bmatrix}
            0 & 0 & 1    
        \end{bmatrix}
        C^{-1}P_{LC}(A) \\
        k & =
        \begin{bmatrix}
            0 & 0 & 1
        \end{bmatrix}
        C^{-1}P_{LC}(A)
    \end{split}
\]

Se define la formula de Ackerman como
\[
    k = 
    \begin{bmatrix}
        0 & 0 & \ldots & 1
    \end{bmatrix}
    C^{-1}P_{LC}(A)
\]