\subsection{Ubicación de polos: metodo directo}
Considere un sistema SISO
\[
    (1)
    \left\{
        \begin{array}{lll}
            \dot{x}(t) = Ax(t) + Bu(t)\\
            y(t) = Cx(t)
        \end{array}
    \right.
\]
El polinomio caracteristico de (1)
\[
    \begin{split}
        P(s) & = det(sI-A) = 0\\
        & = s^{n} + a_{1}s^{n-1} + a_{2}s^{n-2} + \ldots + a_{n}\\
        & = (s-q_{1}) (s-q_{2}) \ldots (s-q_{n})
    \end{split}
\]

donde \(q_{1}, q_{2}, \ldots, q_{n}\) son los polos del sistema en lazo abierto.
El problema de ubicación de polos consiste en asignar los polos \( \mu_{1}, \mu_{2}, \ldots, \mu_{n} \) al sistema en lazo cerrado, entonces
\[
    P_{LC} = (s-\mu_{1}) (s-\mu_{2}) \ldots (s-\mu_{n}) = s^{n} + \tilde{a}_{1}s^{n-1} + \tilde{a}_{2}s^{n-2} + \ldots + \tilde{a}_{n}
\]

Se define la retroalimentación de estado
\[
    \begin{split}
        (2)\;\; 
        u & = r - kx \;\;donde\;\; k:1\times n\\
        & = r - [k_{1}, k_{2}, \ldots, k_{n}]
        \begin{bmatrix}
            x_{1}\\
            x_{2}\\
            \vdots\\
            x_{n}
        \end{bmatrix}\\
        & = r - (k_{1}x_{1} + k_{2}x_{2} + \ldots + k_{n}x_{n})\\
        & = r - \sum_{i=1}^{n} k_{i}x_{i}
    \end{split}
\]

Sustituyendo (2) en (1)

\[
    \begin{split}
        \dot{x} & = Ax + B(r-kx) \\
        & = Ax + Br -Bkx\\
        & = (A-Bk)x + Br \;\; (3) \text{ sistema en lazo cerrado}
    \end{split}
\]
\[
    \begin{split}
        P_{LC}(s) & = det(sI-(A -Bk)) = 0\\
        & = (s-\mu_{1}) (s-\mu_{2}) \ldots (s-\mu_{n})\\
        & = s^{n} + \tilde{a}_{1}s^{n-1} + \tilde{a}_{2}s^{n-2} + \ldots + \tilde{a}_{n}
    \end{split}
\]