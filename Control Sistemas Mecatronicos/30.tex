\subsection{Solución de ecuaciones en espacio de estados}

Considere el sistema
\[
    (1)
    \left\{
        \begin{array}{lll}
            x(k+1)= Ax(k) + Bu(k)\\
            y(k) = Cx(k)
        \end{array}
    \right.
\]
Con condición inicial \( x(0) = x_{0} \)

El problema consiste en determinar el vector de estado \( x(k) \) y la salida \( y(k) \)

Se aplica la transformada \( Z \) al sistema (1)
\[
    \begin{split}
        &\left\{
        \begin{array}{lll}
            \mathcal{Z} \{ x(k+1) \}= \mathcal{Z} \{ Ax(k) \} + \mathcal{Z} \{ Bu(k) \}\\
            \mathcal{Z} \{ y(k) \} = \mathcal{Z} \{ Cx(k) \}
        \end{array}
        \right. \\
        &\left\{
            \begin{array}{lll}
                zX(z) -zX(0) = AX(z) + BU(z) \\
                y(z) = CX(z)
            \end{array}
        \right. \\
        &\left\{
            \begin{array}{lll}
                zX(z) - AX(z) = zX(0) + BU(z) \\
                y(z) = CX(z)
            \end{array}
        \right. \\
        &\left\{
            \begin{array}{lll}
                (zI-A)X(z) = zX(0) + BU(z) \\
                y(z) = CX(z)
            \end{array}
        \right. \\
        (2) \;\; &\left\{
            \begin{array}{lll}
                X(z) = (zI-A)^{-1}zX(0) + (zI-A)^{-1}BU(z) \\
                y(z) = C(zI-A)^{-1}zX(0) + C(zI-A)^{-1}BU(z)
            \end{array}
        \right.
        \end{split}
\]

Para condiciones iniciales \( X(0) = 0 \)
\[
    \frac{Y(z)}{U(z)} = C(zI-A)^{-1}B 
    \left\}
            \begin{array}{lll}
                \text{Matriz de} \\
                \text{transferencia}
            \end{array}
    \right.
\]

Aplicando transformada inversa a (2)
\[
    \mathcal{Z}^{-1} \{ X(z) \} = x(k) = \mathcal{Z}^{-1} \{ (zI-A)^{-1}z \}x(0) + \mathcal{Z}^{-1} \{ (zI-A)^{-1}BU(z) \}
\]
\[
    y(k) = Cx(k) = 
    \underbrace{C \mathcal{Z}^{-1} \{ (zI-A)^{-1}z \}x(0)}_
    {
        \begin{scriptsize}
            \begin{matrix}
                \text{Solución homogénea}\\
                \text{natural}\\
            \end{matrix}
        \end{scriptsize}
    }+ 
    \underbrace{C \mathcal{Z}^{-1} \{ (zI-A)^{-1}BU(z) \}}_
    {
        \begin{scriptsize}
            \begin{matrix}
                \text{Solución particular}\\
                \text{forzada}\\
            \end{matrix}
        \end{scriptsize}
    }
\]

Se define la matriz de transición de estado
\[
    \mathcal{Z}^{-1} \{ (zI-A)^{-1} z \}
\]