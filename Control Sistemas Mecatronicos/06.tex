\subsection{Forma Canonica Controlable}
Ejercicio: demostrar que la propiedad de controlabilidad es invariante para cualquier transformación de similtud

Se considera la transformación de similitud
\[
    (1) X=P^{-1}z
\]

donde \( P \) es invertible
Se toma el sistema 
\[
    (2)
    \left\{
        \begin{array}{lll}
            \dot{x}(t) = Ax(t) + Bu(t)\\
            y(t) = Cx(t)
        \end{array}
    \right.
\]

Se sustituye (1) en (2) para conseguir \( P^{-1}\dot{z} = AP^{-1}z+Bu \), y asi, obtener el sistema
\[
    (3)
    \left\{
        \begin{array}{lll}
            \dot{z}(t) =
                \overbrace{PAP^{-1}}^{\tilde{A}}+ 
                \overbrace{PBu}^{\tilde{B}}\\
            y(t) = CP^{-1}
        \end{array}
    \right.
\]

donde \( A \) y \( PAP^{-1} \) son invariantes, se tiene que \( \lambda (A) = \lambda(PAP^{-1}) \) debido a que \( P \) es invertible, entonces \(rango(\tilde{C}) = rango(C)\).

Se obtiene la matriz de controlabilidad para los sitemas (2) y (3) respectivamente.

\[
    (4)\;\;
    C=[B , AB , A^{2}B , \ldots , A^{n-1}B]
\]
\[
    (5)\;\;
    \tilde{C}=[\tilde{B} , \tilde{A}\tilde{B} , \tilde{A}^{2}\tilde{B} , \ldots , \tilde{A}^{n-1}\tilde{B}]
\]

Considerando que \( (PAP^{-1})^{n} = PA^{n}P^{-1} \), se sustituye las matrices similares de (3) en (5)
\[
    \begin{split}
        \tilde{C} & = [PB, PAP^{-1}PB, PA^{2}P^{-1}PB, \ldots, PA^{n-1}P^{-1}PB] \\
        \tilde{C} & = P[B, AB, A^{2}B, \ldots, A^{n-1}B] \\
        \tilde{C} & = PC 
    \end{split}
\]

Lo que se busca es la matriz \( P \), por lo tanto, de la ecuación anterior se puede despejar de la siguiente forma
\[
    \begin{split}
        \tilde{C} & = PC \\
        \tilde{C}\tilde{C}^{-1} & = PC\tilde{C}^{-1} \\
        I & = PC\tilde{C}^{-1} \\
        P^{-1}I & = C\tilde{C}^{-1} \\
        P^{-1} & = C\tilde{C}^{-1}\;\;(6)
    \end{split}
\]

Ahora se tomará el sistema de entradas y salidas
\[
    \begin{split}
        y^{n}(t) + a_{1}y^{n-1}(t) + \ldots + a_{n}y(t) & = b_{1}u^{n-1}(t) + \ldots + b_{n}u(t) \\
        \mathcal{L} \{ y^{n}(t) + a_{1}y^{n-1}(t) + \ldots + a_{n}y(t) \} & = \mathcal{L} \{ b_{1}u^{n-1}(t) + \ldots + b_{n}u(t) \}\\
        y(s)s^{n} + a_{1}y(s)s^{n-1} + \ldots + a_{n}y(s) & = b_{1}u(s)s^{n-1} + \ldots + b_{n}u(s)\\
        y(s) (s^{n} + a_{1}s^{n-1} +\ldots + a_{n} ) & = u(s) (b_{1}s^{n-1} + \ldots + b_{n})
    \end{split}
\]

Se escribe la función de transferencia como 
\[
    g(s) = \frac{y(s)}{u(s)} = \frac{ b_{1}s^{n-1} + \ldots + b_{n} }{ s^{n} + a_{1}s^{n-1} +\ldots + a_{n} } = \frac{N(s)}{D(s)}
\]

Si consideramos una función de transferencia racional (sin retardo) y estrictamente propia (orden del denominador mayor que el numerador)
\[
    g(s)=N(s)D(s)^{-1}
\]
donde N(s) y D(s) son polinomios. Por lo tanto, la función de transferencia se puede escribir como 
\[
    g(s) = \frac{y(s)}{u(s)} = N(s)D(s)^{-1}
\]

La salida del sistema se puede escribir como 
\[
    y(s) = N(s)D(s)^{-1}u(s)
\]

donde \( v(s) = D(s)^{-1}u(s) \). Entonces la entrada y la salida se pueden escribir como 
\[
    \begin{split}
        u(s) & = D(s)v(s) \\
        y(s) & = N(s)v(s)
    \end{split}
\]

Se definen las variables de estado de la siguiente forma
\[ 
    \begin{bmatrix}
        X_{1}(s) \\
        X_{2}(s) \\
        \vdots \\
        X_{n-1}(s) \\
        X_{n}(s)
    \end{bmatrix}
    =
    \begin{bmatrix}
        S^{n-1} \\
        S^{n-2}\\
        \vdots \\
        S \\
        1
    \end{bmatrix} v(s)
\]

entonces las variables de estado se pueden escribir como

\[
    \begin{split}
        X_{n} & = v(s)\\
        X_{n-1} = sv(s) & = sX_{n}(s)\\
        X_{n-2}(s) = S^{2}v(s) =s(sv(s)) & = sX_{n-1}(s)\\
        &\;\;\vdots\\
        X_{2}(s) & = sX_{3}(s)\\
        X_{1}(s) & = sX_{2}(s)
    \end{split}
\]

Si se sustituyen las variables de estado en la entrada se obtiene lo siguiente
\[
    u(s) = sX_{1}(s) + a_{1}X_{1}(s) + \ldots + a_{n}X_{n}(s)
\]

Se despeja \( sX_{1}(s) \)
\[
    sX_{1}(s) = -a_{1}X_{1}(s) - \ldots - a_{n}X_{n} + u(s)
\]

Entonces es posible escribir el espacio de estados como
\[
    \begin{bmatrix}
        SX_{1}(s) \\
        SX_{2}(s) \\
        \vdots \\
        SX_{n}(s)
    \end{bmatrix}
    =
    \underbrace{
        \begin{bmatrix}
        -a_{1} & -a_{2} & \cdots & -a_{n}\\
        1 & 0 & \cdots & 0 \\
        \vdots & \ddots & \cdots & \vdots \\
        0 & \cdots & 1 & 0
        \end{bmatrix}
                }_{\tilde{A}}
    \begin{bmatrix}
        X_{1}(s) \\
        X_{2}(s) \\
        \vdots \\
        X_{n}(s)
    \end{bmatrix}
    +
    \underbrace{
        \begin{bmatrix}
        1\\
        0\\
        \vdots\\
        0
        \end{bmatrix}
                }_{\tilde{B}}
     u(s)
\]

Según la definición de la matriz similar de controlabilidad
\[
    \tilde{C} = [
    \tilde{B},
    \tilde{A}\tilde{B},
    \tilde{A}^{2}\tilde{B},
    \cdots,
    \tilde{A}^{n-1}\tilde{B} ]
\]

Se usará una matriz de 3*3 para encontrar un patrón
\[
    \tilde{C} = 
    \begin{bmatrix}
        1 & -a_{1} & -a_{1}^{2}-a_{2} \\
        0 & 1 & -a_{1} \\
        0 & 0 & 1
    \end{bmatrix}
\]

La inversa esta dada por
\[
    \tilde{C}^{-1} = 
    \begin{bmatrix}
        1 & a_{1} & a_{2} \\
        0 & 1 & a_{1} \\
        0 & 0 & 1
    \end{bmatrix}
\]

Por lo tanto, la inversa de la matriz similar de controlabilidad se puede expresar como 
\[
    \tilde{C}^{-1} = 
    \begin{bmatrix}
        1 & a_{1} & a_{2} & \cdots & a_{n-1} \\
        0 & 1 & a_{1} & \cdots & a_{n-2}\\
        \vdots & \ddots & \ddots & \ddots & \vdots\\
        0 & 0 & 0 & \ddots & a_{1}\\
        0 & 0 & 0 & \cdots & 1
    \end{bmatrix}
\]

Considerando la matriz de controlabilidad 
\[
    C = [B, AB, A^{2}B, \cdots, A^{n-1}B ]
\]

La ecuación (6) se puede escribir como
\[
    P^{-1} =
    [B, AB, A^{2}B, \cdots, A^{n-1}B ]
    \begin{bmatrix}
        1 & a_{1} & a_{2} & \cdots & a_{n-1} \\
        0 & 1 & a_{1} & \cdots & a_{n-2}\\
        \vdots & \ddots & \ddots & \ddots & \vdots\\
        0 & 0 & 0 & \ddots & a_{1}\\
        0 & 0 & 0 & \cdots & 1
    \end{bmatrix}
\]