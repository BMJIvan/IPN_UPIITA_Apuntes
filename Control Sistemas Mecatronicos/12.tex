\subsection{Ecuación de Lyapunov}
El metodo consiste en calcular \( k \) tal que la matriz \( (A-Bk) \) se le agregan los polos deseados en lazo cerrado. 

Sea \( F \) una matriz con los valores propios deseados en lazo cerrado \( \mu_{1}, \mu_{2}, \ldots, \mu_{n} \).

Se aplica una tranformación de similitud \( P \) tal que 
\[
    A-Bk = PFP^{-1}
\]

Se desconocen \( P \) y \( k \), entonces
\[
    \begin{split}
        A-Bk & = PFP^{-1} \\
        (A-Bk)P & = PFP^{-1}P \\
        AP - B\underbrace{KP}_{\Tilde{k}} & = PF\\
        -B\tilde{k} & = PF-AP\\
        B\tilde{k} & 
        = AP - PF
    \end{split}
\]

La cual es la ecuación de Lyapunov en la forma de Silvestre. 
\subsubsection{Procedimiento para propones \( \tilde{k} \)}

\begin{enumerate}
    \item Construir una matriz F. Se recomienda que sea diagonal por bloques
    \item Seleccionar \(\tilde{k}\), tal que el par \( (F, \tilde{k}) \) sea observable, es decir
        \[
            O = \begin{bmatrix}
                    \tilde{k} \\
                    \tilde{k}F \\
                    \tilde{k}F^{2} \\
                    \vdots \\
                    \tilde{K}F^{n}
                \end{bmatrix}, \;\;  det(O) \not= 0
        \]
    \item La ecuación de Lyapunov es un sistema de ecuaciones lineales, en MATLAB se puede solucionar usando \( P=lyap(A, -F, -B\tilde{k}) \)
    \item Recordando que \( \tilde{k} = kP \), las ganancias se obtienen con 
    \[
            k = \tilde{k}P^{-1}
    \]
\end{enumerate}

La construcción de la matriz \( F \) se realizará con las siguientes reglas
\begin{enumerate}
    \item Polos repetidos: se colocan en la diagonal, pero se agrega un 1
    \item Polos reales no repetidos: se colocan en la diagonal
    \item Polos imaginarios: Los reales se colocan en la diagonal, los imaginarios alternan de signo
\end{enumerate}

Ejemplo: construir la matriz F segun los siguientes polos
\[
    \Big\{-1, -2, -3+4j, -3-4j, -1 \Big\}
\]

Se obtiene lo siguiente
\[
    \begin{bmatrix}
        -1 & 1 & 0 & 0 & 0 \\
        0 & -1 & 0 & 0 & 0 \\
        0 & 0 & 2 & 0 & 0 \\
        0 & 0 & 0 & -3 & -4 \\
        0 & 0 & 0 & 4 & -3
    \end{bmatrix}
\]