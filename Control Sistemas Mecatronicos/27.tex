\subsection{Forma canónica controlable}

considere la representación entrada-salida 
\[
    (1) \;\;
    y(k) + a_{1}y(k-1) + \ldots + a_{n}y(k-n) = b_{1}u(k-1) + \ldots + b_{n}u(k-n)
\]  

Aplicando transformada Z a (1)
\[
    Y(z) + a_{1}z^{-1}Y(z) + \ldots + a_{n}z^{-n}Y(z) = b_{1}z^{-1}U(z) + \ldots + b_{n}z^{-n}U(z)
\]

Se obtiene la función de transferencia
\[
    \begin{split}
        \frac{Y(z)}{U(z)} & = \frac{b_{1}z^{-1} + b_{2}z^{-2} + \ldots + b_{n}z^{-n}}{1 + a_{1}z^{-1} + a_{2}z^{-2} + \ldots + a_{n}z^{-n}} \\ \;\; (2)
        & = \frac{b_{1}z^{n-1} + b_{2}z^{n-2} + \ldots + b_{n}}{z^{n} + a_{1}z^{n-1} + a_{2}z^{n-2} + \ldots + a_{n}} 
    \end{split}
\]

De (2)
\[
    \begin{split}
        & \frac{Y(z)}{b_{1}z^{-1} + b_{2}z^{-2} + \ldots + b_{n}z^{-n}} = \frac{U(z)}{1 + a_{1}z^{-1} + a_{2}z^{-2} + \ldots + a_{n}z^{-n}} = Q(z) \\
        & U(z) = Q(z) + a_{1}z^{-1}Q(z) + a_{2}z^{-2}Q(z) + \ldots + a_{n}z^{-n}Q(z) \\
        & (3)
        \left\{
            \begin{array}{lll}
                Q(z) = - a_{1}z^{-1}Q(z) - a_{2}z^{-2}Q(z) - \ldots - a_{n}z^{-n}Q(z) + U(z) \\
                Y(z) = b_{1}z^{-1}Q(z) + b_{2}z^{-2}Q(z) + \ldots + b_{n}z^{-n}Q(z)
            \end{array}
        \right.
        \end{split}
\]

A partir de (3) se definen las variables de estado
\[
    (4)\;\; X(z) = 
    \begin{bmatrix}
        X_{1}(z) \\
        X_{2}(z) \\
        \vdots \\
        X_{n-1}(z) \\
        X_{n}(z)
    \end{bmatrix} =
    \begin{bmatrix}
        z^{-n} \\
        z^{-n+1} \\
        \vdots \\
        z^{-2} \\
        z^{-1}
    \end{bmatrix} Q(z) \\
\]

Haciendo el cambio de variables en (3)
\[ (5)
    \left\{
        \begin{array}{lll}
            Q(z) = -a_{n}X_{1} - a_{n-1}X_{2} - \ldots - a_{2}X_{n-1}(z) - a_{1}X_{n}(z) + U(z) \\
            Y(z) = b_{1}X_{n}(z) + b_{2}x_{n-1}(z) + \ldots + b_{n}X_{1}(z) \ =
            [b_{n} \; b_{n-1} \; \ldots \; b_{1}] X(z)
        \end{array}
    \right.
\]

Multiplicamos (4) por z

\[ (6)
    \left\{
        \begin{array}{lll}
            zX_{1}(z) = z^{-n+1}Q(z) = X_{2}(z) \\
            zX_{2}(z) = z^{-n+2}Q(z) = X_{3}(z) \\
            \;\;\;\;\;\;\; \vdots \\
            zX_{n-1}(z) = z^{-1}Q(z) = X_{n}(z) \\
            zX_{n}(z) = Q(z)
        \end{array}
    \right.
\]

Se sustituye (4) en (5), considerando el ultimo termino de (6)
\[ (7)
    \left\{
        \begin{array}{lll}
            zX_{n}(z) = Q(z) = [-a_{n} \; -a_{n-1} \; \ldots \; -a_{1}]X(z) + U(z) \\
            Y(z) = [b_{n} \; b_{n-1} \; \ldots \; b_{1}] X(z)
        \end{array}
    \right.
\]

se aplica la transformada inversa a las variables de estado(4)
\[
    X(k) = 
    \begin{bmatrix}
        X_{1}(k) \\
        X_{2}(k) \\
        \vdots \\
        X_{n-1}(k) \\
        X_{n}(k)
    \end{bmatrix}
\]

Se aplica la transformada inversa a (6) y (7)
\[ (7)
    \left\{
        \begin{array}{lll}
            X_{1}(k+1) = X_{2}(k) \\
            X_{2}(k+1) = X_{3}(k) \\
            \;\;\;\;\;\;\; \vdots \\
            X_{n-1}(k+1) = X_{n}(k) \\
            X_{n}(k+1) = Q(k) = [-a_{n} \; -a_{n-1} \; \ldots \; -a_{1}]X(k) + U(k) \\
            y(k) = [b_{n} \; b_{n-1} \; \ldots \; b_{1}] X(k)
        \end{array}
    \right.
\]

se obtiene la forma canónica controlable de (7)
\[
    \begin{split}
        x(k+1) & = 
        \begin{bmatrix}
            0 & 1 & 0 & \cdots & 0 \\
            0 & 0 & 1 & \cdots & 0 \\
            \vdots & \vdots & \ddots & \ddots & \vdots \\
            0 & 0 & \ldots & 0 & 1 \\
            -a_{n} & -a_{n-1} & \cdots & -a_{2} & -a_{1}
        \end{bmatrix}x(k) +
        \begin{bmatrix}
            0 \\ 0 \\ 0 \\ \vdots \\ 1
        \end{bmatrix}u(k) \\
        y(k) & = 
        \begin{bmatrix}
            b_{n} & b_{n-1} & \ldots b_{2} & b_{1}    
        \end{bmatrix} X(k)
    \end{split}
\]