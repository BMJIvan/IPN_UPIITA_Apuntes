\subsection{Estabilidad de Lyapunov}

Método 1 directo \\
Requiere de la solución del sistema

\parskip .1in
Método 2 indirecto \\
No requiere solución del sistema, se basa en la dinámica del sistema

Definición (función definida positiva)
Una función escalar \( V(x) \) es definida positiva si
\[
    \begin{split}
        1) & \;\; V(0) = 0 \\
        2) & \;\; V(x) > 0 \text{  para todo \( x \not = 0 \)}
    \end{split}
\]

Definición (forma cuadrática)\\
Sea una matriz \( p > 0 \) y sea simétrica \( P = P^{T} \)
Con \( V(x) = x^{T}Px > 0 \)
Se tiene la desigualdad de Rayleight
\[
    \begin{split}
        \lambda_{min}(P) ||x||^{2} \leq & V(x) = x^{T}Px \leq \lambda_{max} (P) ||x||^{2} \\
        \lambda_{min}(I)||x||^{2} \leq & V(x) \leq \lambda(I) ||x^{*}||^{2}
    \end{split}
\]

Sea el sistema
\[
    \begin{split}
        & \dot{x}(t) = f(x)\\
        & x(0) = x_{0}
    \end{split}
\]

donde \( x \in R^{n} \) es el vector de estado y \( f: R^{n} \to R^{n}\)

Definición  (punto de equilibrio)
Un punto de equilibrio es aquel que satisface \( f(x^{*}) = 0 \)
\begin{itemize}
    \item a) Tiene un punto de equilibrio: 
    \[
        x^{*} = 0 \text{  Si A es invertible}
    \]
    \item b) Tiene un numero infinito de puntos de equilibrio, si A no es invertible
\end{itemize}

Segundo método de Lyapunov

Sea el sistema
\[
    \begin{split}
        & \dot{x}(t) = f(x)\\
        & f(0) = 0
    \end{split}
\]

Si existe una función \( V(x) \) tal que
\[
    \left\{
        \begin{array}{lll}
            (1) V(x) > 0\\
            (2) \dot{V}(x) \big |_{(x)} < 0
        \end{array}
    \right.
\]
entonces el punto de equilibrio es asintoticamente estable.

Sea el sistema \((1) \;\; \dot{x} = Ax \) determinar las condiciones de estabilidad en el sentido de Lyapunov
\begin{itemize}
    \item 1) 
    \[
        V(x) = x^{T}Px, \;\;\; P = P^{T}, \;\;\; P > 0
    \]
    
    \item 2) 
    \[
        \begin{split}
            \dot{V}(x) & = \dot{x}Px + x^{T}P\dot{x} \;\;\; \text{\;\;\; \;\;\(\dot{x}^{T} = (Ax)^{T} = x^{T}A^{T}\) }\\
            & = x^{T}A^{T}Px + x^{T}P\dot{x} \\
            & = x^{T}(A^{T}P + PA)x \;\;\; \text{  Sea \( Q > 0\) \( A^{T}P + PA = -Q\)} \\
            & = -x^{T}Qx
        \end{split}
    \]
\end{itemize}