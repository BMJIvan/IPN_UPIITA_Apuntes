\section{Solución de ecuaciones en espacio de estado 2}

Se condidera el sistema
\[
    (1)
    \left\{
        \begin{array}{lll}
            \dot{x}(t) = Ax(t) + Bu(t)\\
            y(t) = Cx(t) \;\;\;\;x(0)=X_{0}
        \end{array}
    \right.
\]

\[
    \begin{split}
        \dot{x}(t) -Ax(t) & = Bu(t)\\
        e^{-At} (\dot{x}(t)-Ax(t)) & = e^{-At}Bu(t)\\
        e^{-At} (\dot{x}(t)) - e^{-At}Ax(t) & = e^{-At}Bu(t)\\
        \int_{0}^{t} \frac{d}{dt} (e^{-A\tau}x(\tau)) & = \int_{0}^{t} Bu(\tau)d\tau\\
        e^{-A\tau}\Big|_a^b & = \int_{0}^{t} e^{-A\tau} Bu(\tau)d\tau\\
        e^{-At}x(t) - e^{0}x(0) & = \int_{0}^{t} e^{-A\tau} Bu(\tau)d\tau\\
        e^{At} (e^{-At}x(t) - e^{0}x(0)) & = e^{At} \Big(\int_{0}^{t} e^{-A\tau} Bu(\tau)d\tau\Big)\\
        x(t) & = e^{At}x(0) + \int_{0}^{t} e^{(t-\tau)}Bu(\tau)d\tau + C\\
    \end{split}
\]
anteriormente se obtuvo
\[
    x(t) = \mathcal{L}^{-1} \{ (sI-A)^{-1} \} x(0) + \mathcal{L}^{-1} \{ (sI-A)^{-1}Bu(s) \}
\]
Por lo tanto
\[
    e^{At} = \mathcal{L}^{-1} \{ (sI-A)^{-1} \}
\]
A continuación se desarrollará los dos terminos un poco mas.

Se considera la matriz exponencial de la siguiente forma 

\[
    \begin{matrix}
        f(t)=e^{At} & f(0)=I
    \end{matrix}
\]

entonces, partiendo de la derivada 

\[
    \begin{split}
        \dot{f(t)} & = Ae^{At} = Af(t)\\
        \mathcal{L} \{ \dot{f(t)} \} & = \mathcal{L} \{ Af(t) \}\\
        sF(s)-AF(s) & = AF(s)\\
        sF(s)-AF(s) & = F(0)\\
        (sI-A)F(s) & = I\\
        (sI-A)^{-1} (sI-A)F(s) & = (sI-A)^{-1}I\\
        F(s) & = (sI-A)^{-1}\\
        \mathcal{L}^{-1} \{ F(s) \} & = \mathcal{L}^{-1} \{ (sI-A)^{-1} \}\\
        F(t) & = \mathcal{L}^{-1} \{ (sI-A)^{-1} \}\\
        e^{At} & = \mathcal{L}^{-1} \{ (sI-A)^{-1} \}
    \end{split}
\]

Por otro lado, considerando la definicion de la convolución

\[
    h(t)=(f*g)(t) = \int_{0}^{t} f(t-\tau)g(\tau)d\tau 
\]

Según el teorema de la covolución. si \( \mathcal{L} \{ f(t) \} y \mathcal{L} \{g(t)\} \) existen para \( s>a\geq 0, \) entonces

\[
    \mathcal{L} \{f* g\} = \mathcal{L} \{ f(t) \} \mathcal{L} \{ g(t) \}=F(s)G(s)
\]

considerando las ecuaciones 
\[
    \begin{matrix}
        f(t)=e^{At}, & g(t)=Bu(t)
    \end{matrix}
\]

Se tiene que 
\[
    \int_{0}^{t} e^{At-\tau} Bu(\tau)d\tau = f*g
\]

Aplicando la transformada de Laplace se tiene que

\[
    \begin{split}
        \mathcal{L} \Big\{ \int_{0}^{t} e^{At-\tau} Bu(\tau)d\tau \Big\} & =\mathcal{L} \Big\{ f*g \Big\}\\
        & = F(s)G(s)\\
        & = (SI-A)^{-1}Bu(s)
    \end{split}
\]

Por lo tanto aplucando la transformada inversa de Laplace se obtiene

\[
    \int_{0}^{t} e^{At-\tau} Bu(\tau)d\tau = \mathcal{L}^{-1} \Big\{ (SI-A)^{-1}Bu(s) \Big\}
\]