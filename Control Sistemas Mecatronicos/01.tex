\section{UNIDAD I}
\subsection{Solución de ecuaciones en espacio de estado}

\[
    (1)
    \left\{
        \begin{array}{lll}
            \dot{x}(t) = Ax(t) + Bu(t)\\
            y(t) = Cx(t)
        \end{array}
    \right.
\]

con condición inicial
\[ x(0) = x_{0}.\]
\[
A:n\times n,\;\; 
x:n\times 1,\;\;
B:n\times m,\;\;
u:m\times 1,\;\;
C:p\times n,\;\;
y:p\times 1,\;\;
\]

El problema consiste en determinar \(x(t)\) y la respuesta \(y(t)\)

Aplicando la transformada de Laplace a (1)
\[
\begin{split}
    \mathcal{L} \{ \dot{x}(t) \} & = \mathcal{L} \{ Ax(t) \} + \mathcal{L} \{ Bu(t) \} \\
    sX(s) - X(0) & = AX(s) + Bu(s)\\
    \underbrace{sX(s)}_{n\times1} - \underbrace{AX(s)}_{n\times1} & = X(0) + Bu(s)\\
    (sI-A)X(s) & = X(0) + Bu(s) \\
    X(s) & = (sI-A)^{-1}X(0) + (sI-A)^{-1}Bu(s) \\
    \mathcal{L}^{-1} \{ X(s) \} = x(t) & = \mathcal{L}^{-1} \{ (sI-A)^{-1} \} x(0) + \mathcal{L}^{-1} \{ (sI-A)^{-1}Bu(s) \} \\
    y(t) & = 
        \underbrace{C\mathcal{L}^{-1} \{ (sI-A)^{-1} \}X(0)}_
        {
        \begin{scriptsize}
            \begin{matrix}
                \text{Solución homogenea}\\
                \text{Respuesta en estado transitorio}\\
                \text{Respuesta natural}\\
            \end{matrix}
        \end{scriptsize}
        }+
        \underbrace{C\mathcal{L}^{-1} \{ (sI-A)^{-1}Bu(s) \}}_
        {
        \begin{scriptsize}
            \begin{matrix}
                \text{Solución particular}\\
                \text{Respuesta en estado estacionario}\\
                \text{Respuesta forzada}\\
            \end{matrix}
        \end{scriptsize}
        }
\end{split}
\]