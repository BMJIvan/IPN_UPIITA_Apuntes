\subsection{Diseño de observadores de estado: forma canonica observable}

Considere el sistema SISO
\[
    (1)
    \left\{
        \begin{array}{lll}
            \dot{x} = Ax + Bu\\
            y = Cx
        \end{array}
    \right.
\]

Se asume que el sistema (1) es observable, y el polinomio caracteristico del sistema es 
\[
    \begin{split}
        P(s) & = det(sI-A) \\
        & = (s-q_{1}) ( s-q_{2}) \ldots (s-q_{n})\\
        & = s^{n} + a_{1}s^{n-1} + a_{2}s^{n-2} + \ldots + a_{n}
    \end{split}
\]

Haciendo \( z=xQ^{-1} \) o bien \( x=Qz \;\; (2)\) donde 
\[
    Q^{-1} =
    \begin{bmatrix}
    a_{n-1} & a_{n-2} & \cdots & a_{1} & 1 \\
    a_{n-2} & a_{n-3} & \cdots & 1 & 0\\
    \vdots & \rotatebox{90}{\(\ddots\)} & \rotatebox{90}{\(\ddots\)} & \vdots & \vdots\\
    a_{1} & 1 & 0 & \cdots & 0\\
    1 & 0 & 0 & \cdots & 0
    \end{bmatrix}
    \begin{bmatrix}
        C\\
        CA\\
        CA^{2}\\
        \vdots\\
        CA^{n-1}
    \end{bmatrix}
\]

Sustituyendo (2) en (1) se tiene que
\[
    \begin{split}
        Q\dot{z} & = AQz + Bu\\
        \dot{z} & = \underbrace{Q^{-1}AQ}_{\tilde{A}}z + \underbrace{Q^{-1}B}_{\tilde{B}}u\\
        \dot{z} & = 
            \begin{bmatrix}
                0 & \cdots & 0 & -a_{n} \\
                1 & \cdots & 0 & -a_{n-1} \\
                \vdots & \ddots & \vdots & \vdots \\
                0 & \ldots & 1 & -a_{1}
            \end{bmatrix} z +
            \begin{bmatrix}
                b_{n} \\ b_{n-1} \\ \vdots \\ b_{1}
            \end{bmatrix} u
    \end{split}
\]
\[
    \begin{split}
        y & = \underbrace{CQ}_{\tilde{C}}z \\
        & = \begin{bmatrix}
                0 & 0 & \ldots & 1
            \end{bmatrix} z
    \end{split}
\]

Se tiene el observador de estado
\[
    \begin{split}
        (3) \;\; \dot{\hat{x}} & = A\hat{x} + Bu + L(y-C\hat{x}) \\
        & = A\hat{x} + Bu + Ly - LC\hat{x} \\
        & = A\hat{x} - LC\hat{x} +Bu + Ly\\
        & = (A-LC)\hat{x} + \begin{bmatrix}B & L \end{bmatrix}\begin{bmatrix}   u\\ y \end{bmatrix}
    \end{split}
\]

Sustituyendo (2) en (3) se tiene que
\[
    \begin{split}
        Q\dot{\hat{z}} & = AQ\hat{z} + Bu + L(y-CQ\hat{z}) \\
        \dot{\hat{z}} & = Q^{-1}AQ\hat{z} + Q^{-1}Bu + \underbrace{Q^{-1}L}_{\tilde{L}}(y-CQ\hat{z}) \\
        \dot{\hat{z}} & = Q^{-1}AQ\hat{z} + Q^{-1}Bu + Q^{-1}Ly - Q^{-1}LCQ\hat{z} \\
        \dot{\hat{z}} & = Q^{-1}(A-LC)Q\hat{z} + Q^{-1}(Bu+Ly)
    \end{split}
\]
Donde \( \tilde{L} = Q^{-1}L \) (4)

Ya que \( A-LC \) y \(Q^{-1}(A-LC)Q\) son similares, se tiene que
\[
    \begin{split}
        P_{obs} & = det(sI-(A-LC)) = det(sI-Q^{-1}(A-LC)Q)\\
        & = (s-\mu_{1}) (s-\mu_{2}) \ldots (s-\mu_{n})\\
        & = s^{n} +\tilde{a}_{1}s^{n-1} + \tilde{a}_{2}s^{n-2} + \ldots + \tilde{a}_{n}
    \end{split}
\]

Si \( \tilde{A} \) se obtiene del polinomio caracteristico del sistema \( A \), entonces se puede escribir el sistema similar del observador \( \tilde{A}-\tilde{L}\tilde{C} \) de la siguiente forma
\[
    \tilde{A}-\tilde{L}\tilde{C}=
    \begin{bmatrix}
        0 & 0 & \cdots & -\tilde{a}_{n} \\
        1 & 0 & \cdots & -\tilde{a}_{n-1} \\
        \vdots & \ddots & \ddots & \vdots \\
        0 & \cdots & 1 & -\tilde{a}_{1}
    \end{bmatrix} = \tilde{A}^{'}
\]

El problema consiste en despejar las ganancias \( \tilde{L}_{1}, \tilde{L}_{2}, \ldots, \tilde{L}_{n}  \)
\[
    \begin{split}
        \tilde{A} - \tilde{L} \tilde{C} & = \tilde{A}^{'} \\ 
    -\tilde{L} \tilde{C} & = \tilde{A}^{'} - \tilde{A} \\
    \tilde{L} \tilde{C} & = \tilde{A} - \tilde{A}^{'}\\
    \begin{bmatrix}
        \tilde{L}_{1} \\ \tilde{L}_{2} \\ \vdots \\ \tilde{L}_{n}
    \end{bmatrix}
    \begin{bmatrix}
        0 & 0 & \ldots & 1
    \end{bmatrix} & =
    \begin{bmatrix}
        0 & 0 & \cdots & \tilde{a}_{n} \\
        -1 & 0 & \cdots & \tilde{a}_{n-1} \\
        \vdots & \ddots & \ddots & \vdots \\
        0 & \cdots & -1 & \tilde{a}_{1}
    \end{bmatrix}
    +
    \begin{bmatrix}
        0 & 0 & \cdots & -a_{n} \\
        1 & 0 & \cdots & -a_{n-1} \\
        \vdots & \ddots & \ddots & \vdots \\
        0 & \cdots & 1 & -a_{1}
    \end{bmatrix} \\
    \begin{bmatrix}
        0 & 0 & \ldots & \tilde{L}_{1} \\
        0 & 0 & \cdots & \tilde{L}_{2} \\
        \vdots & \ddots & \ddots & \vdots \\
        0 & \cdots & 0 & \tilde{L}_{n}
    \end{bmatrix} 
    & =
    \begin{bmatrix}
        0 & 0 & \cdots & \tilde{a}_{n} - a_{n} \\
        0 & 0 & \cdots & \tilde{a}_{n-1} - a_{n-1} \\
        \vdots & \ddots & \ddots & \vdots \\
        0 & \cdots & 0 & \tilde{a}_{1} - a_{1}
    \end{bmatrix} \\
    \end{split}
\]

El vector de ganacias similares se puede escribir como
\[
    \begin{bmatrix}
        \tilde{L}_{1} \\ \tilde{L}_{2} \\ \vdots \\ \tilde{L}_{n}
    \end{bmatrix}
    =
    \begin{bmatrix}
        \tilde{a}_{n} - a_{n} \\
        \tilde{a}_{n-1} -a_{n-1} \\
        \vdots \\
        \tilde{a}_{1} -a_{1}
    \end{bmatrix}
\]

De la ecuación (4) se puede despajar las ganancias del observador
\[
    \begin{split}
        \tilde{L} & = Q^{-1}L \\
        L & = Q\tilde{L}
    \end{split}
\]