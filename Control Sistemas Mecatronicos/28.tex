\subsection{Forma canónica observable}

Considere el sistema (1)
\[
    \begin{split}
        (1) \;\; & y(k) + a_{1}y(k-1) + a_{2}y(k-2) + \ldots + a_{n}y(k-n) \\
            & = b_{1}u(k-n) + b_{2}u(k-2) + \ldots + b_{n}u(k-n)
    \end{split}
\]

Aplicando transformada Z a la ecuación (1)
\[
    \begin{split}
        & Y(z) + a_{1}z^{-1}Y(Z) + a_{2}z^{-2}Y(z) + \ldots + a_{n}z^{-n}Y(z) \\
        & = b_{1}z^{-1}U(z) + b_{2}z^{-2}U(z) + \ldots + b_{n}z^{-n}U(z) \\
        & Y(z) = z^{-1}[b_{1}U(z)-a_{1}Y(z)] + z^{-2}[b_{2}U(z)-a_{2}Y(z)] + \ldots + \\
        & z^{-n+1}[b_{n-1}U(z)-a_{n-1}Y(z)] + \ldots + z^{-n}[b_{n}U(z)-a_{n}Y(z)] \\
        & Y(z) = z^{-1}[b_{1}U(z)-a_{1}Y(z)] + z^{-1}z^{-1}[b_{2}U(z)-a_{2}Y(z)] + \ldots + \\
        & z^{-n+1}[b_{n-1}U(z)-a_{n-1}Y(z)] + \ldots + z^{-n}[b_{n}U(z)-a_{n}Y(z)] \\
        (2) & Y(z) = z^{-1} [b_{1}U(z)-a_{1}Y(z) + z^{-1}\{b_{2}U(z)-a_{2}Y(z) + z^{-1}(b_{3}U(z)-a_{3}Y(z) + \ldots + \\
        & z^{-1}[b_{n-1}U(z)-a_{n-1}Y(z) + z^{-1}(b_{n}U(z)-a_{n}Y(z))]\ldots)\}]
    \end{split}
\]

Con \( Y(z) = x_{n}(z) \), a partir de (2) se definen las variables de estado
\[(3)
    \left\{
        \begin{array}{lll}
            X_{1}(z) = z^{-1}(b_{n}U(z)-a_{n}Y(z)) \\ 
            x_{2}(z) = z^{-1}(b_{n-1}U(z)-a_{n-1}Y(z)+X_{1}(z)) \\
            \;\;\;\;\;\;\;\;\;\; \vdots \\
            x_{n-1}(z) = z^{-1}(b_{2}U(z)-a_{2}Y(z) + X_{n-2}(z)) \\
            X_{n}(z) = z^{-1}(b_{1}U(z)-a_{1}X_{n}(z)+x_{n-1}(z))
        \end{array}
    \right.
\]

Se multiplica (3) por z y se agrega la salida del sistema
\[(4)
    \left\{
        \begin{array}{lll}
            zX_{1}(z) = b_{n}U(z)-a_{n}Y(z) \\ 
            zx_{2}(z) = b_{n-1}U(z)-a_{n-1}Y(z)+X_{1}(z) \\
            \;\;\;\;\;\;\;\;\;\; \vdots \\
            zx_{n-1}(z) = b_{2}U(z)-a_{2}Y(z) + X_{n-2}(z) \\
            zX_{n}(z) = b_{1}U(z)-a_{1}X_{n}(z)+x_{n-1}(z) \\
            Y(z) = x_{n}(z)
        \end{array}
    \right.
\]

Aplicando transformada inversa  a (4)
\[(5)
    \left\{
        \begin{array}{lll}
            X_{1}(k+1) = b_{n}u(k)-a_{n}y(k) \\ 
            X_{2}(k+1) = b_{n-1}u(k)-a_{n-1}y(k)+X_{1}(k) \\
            \;\;\;\;\;\;\;\;\;\; \vdots \\
            X_{n-1}(k+1) = b_{2}u(k)-a_{2}y(k) + X_{n-2}(k) \\
            X_{n}(k+1) = b_{1}u(k)-a_{1}X_{n}(k)+x_{n-1}(k) \\
            y(k) = X_{n}(k)
        \end{array}
    \right.
\]

A partir de (5) se obtiene la forma canónica observable
\[
    \begin{split}
        x(k+1) & = 
        \begin{bmatrix}
            0 & 0 & 0 & \cdots & -a_{n} \\
            1 & 0 & 0 & \cdots & -a_{n-1} \\
            \vdots & \vdots & \ddots & \ddots & \vdots \\
            0 & 0 & \ldots & 0 & -a_{2} \\
            0 & 0 & \cdots & 1 & -a_{1}
        \end{bmatrix}x(k) +
        \begin{bmatrix}
            b_{n} \\ b_{n-1} \\ \vdots \\ b_{2} \\ b_{1}
        \end{bmatrix}u(k) \\
        y(k) & = 
        \begin{bmatrix}
            0 & 0 & \ldots & 0 & 1    
        \end{bmatrix} X(k)
    \end{split}
\]