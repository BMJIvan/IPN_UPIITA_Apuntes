\subsection{Observadores de estado (reconstruir variables que no se conocen)}

La retroalimentación de estado \( u = r - kx \) asume que todo el vector de estados es conocido \( x = [ x_{1}, x_{2}, \ldots, x_{n} ] \) y \( C=I \) para tener en la salida todas las variables de estado.

\[
    (1)
    \left\{
        \begin{array}{lll}
            \dot{x}(t) = Ax(t) + Bu(t)\\
            y(t) = Cx(t)
        \end{array}
    \right.
\]
%Figura
Sin embargo no siempre se tiene acceso a todas las variables de estado, debido a restricciones tecnologicas, de costos, etc,.

Los obervadores de estado, se utilizan para aproximar el valor de las variables de estado desconocidas.

Sea \( x \) y \( \hat{x} \), donde \( \hat{x} \) es un valor aproximado del vector \( x \) obtenido mediante un observador de estado, entonces \( u = r - k\hat{x} \)
%Figura
Se propone el observador de estado para (1)
%Figura del observador
\[
    \begin{split}
        \dot{\hat{x}} & = 
        \underbrace{A\hat{x} + Bu}_{
            \begin{matrix}
                \text{copia del} \\
                \text{del sistema}
            \end{matrix}}
            +
            \underbrace{L(y-C\hat{x})}_{
            \begin{matrix}
                \text{factor de} \\
                \text{correción}
            \end{matrix}
            }\\
            & = A\hat{x} + Bu + Ly - LC\hat{x}\\
    \end{split}
\]

donde 
\[
    L = 
    \begin{bmatrix}
        L_{1} \\
        L_{2} \\
        \vdots \\
        L_{n}
    \end{bmatrix}
\]
%Figura \hat(x(0)) -> x(0)
%Figura sistema con observador

Convergencia asintotica
\[
    ||x - \hat{x}||\underset{t \to \inf}{\to} 0
\]
\[
    ||x-\hat{x}|| \le k_{0} e^{-\lambda t}
\]

Se define el error de estimación 
\[
    (2) \;\; e(t) = x(t) -\hat{x}(t)
\]

Derivando (2) con repecto al tiempo
\[
    \begin{split}
        \dot{e}(t) & = \dot{x}(t) -\dot{\hat{x}}(t) \\
        & = Ax + Bu -A\hat{x} - Bu - L(y - C \hat{x}) \\
        & = Ax-A\hat{x}-L(Cx -C\hat{x})\\
        & = A
        \underbrace{(x-\hat{x})}_{e(t)} 
        - LC
        \underbrace{(x-\hat{x})}_{e(t)}\\
        & = Ae(t) - LCe(t) \\
        & = (A-LC)e(t) \;\; (3)
    \end{split}
\]

Para resolver la ecuación diferencial (3), se puede ver que \(f(t) = e(t)\) con condición inicial \(f(0) = e(0) \), y la derivada \( \dot{f(t)} = (A-LC)e(t) \). Asi que se propone que \( f(t) =Q_{0}e^{Qt} \), por lo tanto se tiene que
\[
    f(0) = e(0) = Q_{0}e^{Q(0)}=Q_{0}e^{0} = Q_{o}I = Q_{0}
\]

La derivada de la función propuesta es 
\[
    \dot{f(t)} = Q_{0}Qe^{Qt}
\]

igualando funciones
\[
    \begin{split}
        Q_{0}Qe^{Qt} & = (A-LC)e(t) \\
        Q_{0}Qe^{Qt} & = (A-LC)Q_{0}e^{Qt}\\
        Q_{0}Qe^{Qt}(Q_{0}e^{Qt})^{-1} & = (A-LC)Q_{0}e^{Qt}(Q_{0}e^{Qt})^{-1}\\
        Q & = (A-LC) 
    \end{split}
\]

Sustituyendo \( Q \) y \( Q_{0} \) en la función propuesta 
\[
    f(t) = e(0)e^{(A-LC)t}
\]

El problema de diseño de observadores se resuelve como un problema de ubicación de polos, es decir, consiste en calcular \( L \) para asignar dinamica del observador de estado

\[
    \begin{split}
        P_{\text{obs}} & = det(sI-(A-LC)) \\
        & = (s-\mu_{1})(s-\mu_{2})\ldots(s-\mu_{n}) \\
        & = s^{n} + \tilde{a}_{1}s^{n-1} + \tilde{a}_{2}s^{n-2} + \ldots + \tilde{a}_{n} 
    \end{split}
\]