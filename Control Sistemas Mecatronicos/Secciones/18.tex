\subsection{Detectabilidad de sistemas lineales}

Considere el sistema
\[(1)
    \left\{
        \begin{array}{lll}
            \dot{x}(t) = Ax(t) + Bu(t) \\
            y(t) = Cx(t)
        \end{array}
    \right. \;\;
\]

donde 
\[
     \begin{split}
        A &: n \times n \\
        B &: n \times m \\
        C &: p \times n \\
        X &: n \times 1 \\
        y &: p \times 1 \\
        u &: m \times 1 \\
        m &: entradas \\
        p &: salidas
    \end{split}
\]

Definición detectabilidad: El sistema (1) es detectable si existe L tal que la matriz \( A-LC \) es Horwitz estable, es decir, los polos son negativos

\begin{enumerate}
    \item Si el par \( (A, B) \) es observable entonces el sistema (1) siempre es detectable
    \[
    \begin{split}
        O = 
        \begin{bmatrix}
            C & CA & CA^{2} & \ldots & CA^{n-1}
        \end{bmatrix}^{T}
         \;\; rango(O)=n
    \end{split}
    \]
    
    \item Si el par \( (A, C) \) no es observable, es decir, \( rango(C)=r<n \), donde \( r \) es el número de vectores linealmente independientes. 
\end{enumerate}

Se definen \( s_{1}, s_{2}, \ldots, s_{r} \) los primeros vectores fila linealmente independientes de la matriz \( O \) y \( s_{r+1}, \ldots, s_{n} \) vectores fila arbitrarios (se proponen) tales que la matriz \( S \) sea invertible
\[
    S = \begin{bmatrix}
            s_{1} & s_{2} & \ldots & s_{r} & s_{r+1} & \ldots & s_{n}
        \end{bmatrix}^{T}:n \times n \;\; det(S) \not = 0
\]

Entonces la transformación de estado \( z = Sx \) o \( x = S^{-1}z \) es tal que
\[
    \begin{split}
        (2) \;\; S^{-1}\dot{z} & = AS^{-1}z + Bu \\
        \dot{z} & =SAS^{-1} + SBu \\
        & = \begin{bmatrix}
                A_{11} & 0 \\
                A_{21} & A_{22}
            \end{bmatrix} z
            +
            \begin{bmatrix}
                B_{1} \\ B_{2}
            \end{bmatrix} u
    \end{split}
\]

En la salida
\[
    y = CS^{-1} = 
        \begin{bmatrix}
            C_{1} & 0    
        \end{bmatrix} z
\]

Donde 
\[
    \begin{split}
        A_{11} & : r \times r \\
        0 & : r \times (n-r) \\
        A_{21} & : (n-r) \times r \\
        A_{22} & : (n-r) \times (n-r) \\
        C_{1} & : P \times (n-r)
    \end{split} 
\]

El sistema (1) y (2) comparten valores propios, \( \lambda(A) = \lambda(A_{11}) \cup \lambda(A_{22}) \)

Se propone el observador para el sistema (2)
\[
    \begin{split}
        \dot{\hat{z}} & = 
        \begin{bmatrix}
            A_{11} & 0 \\ 
            A_{21} & A_{22}
        \end{bmatrix} \hat{z} +
        \begin{bmatrix}
            B_{1} \\ B_{2}
        \end{bmatrix} u +
        \underbrace{
            \begin{bmatrix}
                \tilde{L}_{1} \\ \tilde{L}_{2}
            \end{bmatrix}
        }_{\tilde{L}}
        (y - \begin{bmatrix}
                C_{1} & 0
             \end{bmatrix} \hat{z} ) \\
        & = 
        \begin{bmatrix}
            A_{11} & 0 \\ 
            A_{21} & A_{22}
        \end{bmatrix} \hat{z} +
        \begin{bmatrix}
            B_{1} \\ B_{2}
        \end{bmatrix} u +
        \begin{bmatrix}
            \tilde{L}_{1} \\ \tilde{L}_{2}
        \end{bmatrix} y -
        \begin{bmatrix}
            \tilde{L}_{1} \\ \tilde{L}_{2}
        \end{bmatrix}
        \begin{bmatrix}
            C_{1} & 0
        \end{bmatrix} \\
        & = 
        \begin{bmatrix}
            A_{11} & 0 \\ 
            A_{21} & A_{22}
        \end{bmatrix} \hat{z} +
        \begin{bmatrix}
            B_{1} \\ B_{2}
        \end{bmatrix} u +
        \begin{bmatrix}
            \tilde{L}_{1} \\ \tilde{L}_{2}
        \end{bmatrix} y -
        \begin{bmatrix}
            \tilde{L}_{1}C_{1} & 0 \\
            \tilde{L}_{2}C_{1} & 0
        \end{bmatrix} \\
        & = 
        \begin{bmatrix}
            A_{11}-\tilde{L}_{1}C_{1} & 0 \\
            A_{21}-\tilde{L}_{2}C_{1} & A_{22}
        \end{bmatrix} \hat{z} +
        \begin{bmatrix}
            B_{1} \\ B_{2}
        \end{bmatrix} u
        +
        \begin{bmatrix}
            \tilde{L}_{1} \\ \tilde{L}_{2}
        \end{bmatrix} y \\
    \end{split}
\]

La dinámica del observador es
\[
    \lambda(A_{11} - \tilde{L}_{1}C_{1}) \cup \lambda(A_{22})
\]

Una ecuación necesaria para detectabilidad es que \( A_{22} \) sea Hurwits estable

La ganancia \( \tilde{L}_{1} \) se calcula con los métodos de diseño de observadores para la matriz \( A_{11}-\tilde{L}_{1}C_{1} \) donde \( \tilde{L}_{2} \) es arbitrario, por lo tanto se propone \( \tilde{L}_{2} = 0 \)
\[
    \tilde{L} = 
    \begin{bmatrix}
        \tilde{L}_{1} \\ 0    
    \end{bmatrix}
\]

Aplicando la transformación de estado \( x = S^{-1}z \) al observador del sistema (1) 
\[
    \begin{split}
        \hat{x} & = A\hat{x} + Bu + L(y-C\hat{x}) \\
        S^{-1}\dot{\hat{z}} & = AS^{-1}\hat{z} + Bu + L(y-CS^{-1}\hat{z}) \\
        \dot{\hat{z}} & = SAS^{-1}\hat{z} + SBu + SL (y-CS^{-1}\hat{z})
    \end{split}
\]
se define \( \tilde{L} = SL \)

Las ganancias del observador se obtienen con 
\[
    L = S^{-1}\tilde{L}
\]