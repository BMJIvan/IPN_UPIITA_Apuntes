\subsection{Criterios de controlabilidad}

El sistema 
\[(1)
    \left\{
        \begin{array}{lll}
            \dot{x}(t) = Ax(t) + Bu(t) \\
            y(t) = Cx(t)
        \end{array}
    \right.
\]

O el par \( (A, B)\) es controlable si cumple alguno de los siguientes criterios.

1) Controlabilidad de Kalman
La matriz de controlabilidad
\[
    C = 
    \begin{bmatrix}
        \underbrace{B}_{n\times m} &
        \underbrace{AB}_{n\times m} &
        A^{2}B & \ldots & A^{n-1}B
    \end{bmatrix}
    \Big\} n \times nm
\]

para \( m=1 \) (una entrada), con \( det(C)\not= 0 \) es de rango completo, es decir, \( rango(C)=n \).

2) Controlabilidad de Hautus
La matriz de controlabilidad
\[
    H =
    \begin{bmatrix}
        \lambda I-A & B
    \end{bmatrix}
    \Big \} n \times (n+m)
\]
es de rango completa, \( rango(HC)=n \) para todo \( \lambda \in \mathbb{C} \)

3) Gramiano de controlabilidad
\[
    G_{c} = \int_{0}^{t} e^{At\tau}BB^{T}e^{A^{T}\tau}d\tau
\]
es invertible, es decir, \( det(G_{c}) \not= 0 \)

4) Los valores propios de la matriz \( A-Bk \) pueden asignarse arbitrariamente.

\subsection{Criterios de observabilidad}
El sistema 
\[
    (1)
    \left\{
        \begin{array}{lll}
            \dot{x}(t) = Ax(t) + Bu(t) \\
            y(t) = Cx(t)
        \end{array}
    \right.
\]

O el par \( (A, B)\) es observable si cumple alguno de los siguientes criterios.

1) Observabilidad de Kalman
La matriz de observabilidad
\[
    C = 
    \left.
        \begin{bmatrix}
            C \\ CA \\ CA^{2} \\ \vdots \\ CA^{n-1}
        \end{bmatrix}
    \right\} pn\times n
\]
es de rango completo, \( rango(C)=n \).

2)Observabilidad de Hautus
La matriz de observabilidad
\[
    H_{o} =
    \begin{bmatrix}
        \lambda I-A \\ C
    \end{bmatrix}
\]
es de rango completa, \( rango(H_{o})=n \) para todo \( \lambda \in \mathbb{C} \)

3) Gramiano de observabilidad
\[
    G_{o} = \int_{0}^{t} e^{At\tau}C^{T}Ce^{A^{T}\tau}d\tau
\]
es invertible, \( det(G_{o}) \not= 0 \)

4) Los valores propios de la matriz \( A-LC \) pueden asignarse arbitrariamente.