\subsection{Estabilización por retroalimentación de estado}

Considere el sistema
\[(1)
    \left\{
        \begin{array}{lll}
            \dot{x}(t) = Ax(t) + Bu(t) \\
            y(t) = Cx(t)
        \end{array}
    \right. \;\;
\]

donde 
\[
     \begin{split}
        A &: n\times n \\
        B &: n\times m \\
        C &: p \times 1 \\
        \underset{m \times n}{k} & = 
            \begin{bmatrix}
                k_{11} & k_{12} & \cdots & k_{1n} \\
                k_{21} & k_{22} & \cdots & k_{2n} \\
                \vdots & \vdots & \vdots & \vdots \\
                k_{n1} & k_{n2} & \cdots & k_{nn}
            \end{bmatrix}
    \end{split}
\]

Definición estabilización: El sistema (1) es estable por retroalimentación de estado si existe \( u = -kx \) tal que el sistema en lazo cerrado
\[
    \dot{X} = (A-Bk)x \;\; \text{es estable} 
\]

es decir, que todos los valores característicos de la matriz \( A-Bk \) tienen parte real negativa

Se tienen dos casos 
\begin{enumerate}
    \item Si el par \( (A, B) \) es controlable entonces el sistema (1) siempre es estabilizable por retroalimentación de estado
    \[
        C = \begin{bmatrix}
                B & AB & A^{2}B & \ldots & A^{n-1}B
            \end{bmatrix} \;\; rango(C) = n
    \]
    \item Si el par \( (A, B) \) no es controlable, \( rango(C) = r < n \), donde \( r \) es el numero de vectores linealmente independientes. 
    En este caso se debe determinar las condiciones para calcular la ganancia k
\end{enumerate}

Teorema: sea la matriz de controlabilidad
\[
    C = \begin{bmatrix}
            B & AB & A^{2}B & \ldots & A^{n-1}B
        \end{bmatrix} \;\; rango(C) = n
\]

Se definen \( q_{1}, q_{2}, \ldots, q_{r} \) vectores columna de la matriz \( C \) linealmente independientes, y \( q_{r+1}, q_{r+2}, \ldots, q_{n} \) vectores columna arbitrarios (se proponen) linealmente independientes tales que la matriz \( Q \) sea invertible
\[
    Q = \begin{bmatrix}
            q_{1} & q_{2} & \ldots & q_{r} & q_{r+1} & \ldots & q_{n} 
        \end{bmatrix}
\]

Entonces la transformación \( x = Qz \) es tal que
\[  
    \begin{split}
        (2) \;\; Q\dot{z} & = AQz + Bu \\
        z & = Q^{-1}AQz + Q^{-1}Bu \\
        & = \underbrace{
                \begin{bmatrix}
                    A_{11} & A_{12} \\
                    0 & A_{22}
                \end{bmatrix}
                        }_
                        {
                \begin{matrix}
                    \text{Descomposición \; de \; Kalman} \\
                     \text{(controlable, no controlable)}
                \end{matrix} 
                        }
            z +
            \begin{bmatrix}
                B_{1} \\ 0
            \end{bmatrix} u
    \end{split}
\]

Donde
\[
    \begin{split}
        A_{11} & : r \times r \\
        A_{12} & : r \times (n-r) \\
        O & : (n-r) \times r \\
        B_{1} & : r \times m \\
        A_{22} & : (n-r) \times (n-r)
    \end{split}
\]

El sistema (1) y (2) son similares, es decir \( \lambda(A) = \lambda(A_{11}) \cup \lambda(A_{22}) \)

Se define 
\[
    (3) \;\; u = \tilde{k}z
\]

Donde \( \tilde{k} = kQ^{-1} \)

Sustituyendo (3)  en (2)
\[
    \begin{split}
        \dot{z} & = 
        \begin{bmatrix}
            A_{11} & A_{12} \\
            0 & A_{12}
        \end{bmatrix} z
        -
        \begin{bmatrix}
            B_{1} \\ 0
        \end{bmatrix}
        \begin{bmatrix}
            \tilde{k}_{1} & \tilde{k}_{2}
        \end{bmatrix} z \\
        & = 
        \begin{bmatrix}
            A_{11} & A_{12} \\
            0 & A_{12}
        \end{bmatrix} z
        -
        \begin{bmatrix}
            B_{1} \tilde{k}_{1} & B_{1} \tilde{k}_{2} \\
            0 & 0
        \end{bmatrix} z
    \end{split}
\]

Los valores característicos del sistema en lazo cerrado son 
\[
    \underbrace{ \lambda(A_{11} - B_{1}\tilde{k}_{1}) }_
        {
        \begin{matrix}
            \text{polos} \\
            \text{controlables}
        \end{matrix}
        } \;\;\;\cup
    \underbrace{ \lambda(A_{22}) }_
        {
        \begin{matrix}
            \text{polos no} \\
            \text{controlables}
        \end{matrix}
        }
\]

Condición necesaria para estabilización es que la matriz \( A_{22} \) se estable

El problema de estabilización se resuelve como un problema de ubicación de polos para la matriz \( A_{11}-B_{1}\tilde{k}_{1} \)

donde 
\[
    \begin{split}
        \tilde{k} = 
        \begin{bmatrix}
            \tilde{k}_{1} & \Vec{0}
        \end{bmatrix}
    \end{split}
\]

Las ganancias en lazo cerrado se obtienen con
\[
    k = \tilde{k} \tilde{Q}^{-1}
\]