\subsection{Forma Canonica Observable}
Ejercicio: Demostrar que la propiedad de observabilidad es invariante para cualquier transformación de similitud.

Se considera la transformación de similitud
\[
    (1)\:\:
    x = Qz
\]

donde Q es invertible. Se toma el sistema
\[
    (2)
    \left\{
        \begin{array}{lll}
            \dot{x}(t) = Ax(t) + Bu(t)\\
            y(t) = Cx(t)
        \end{array}
    \right.
\]

Se sustituye (1) en (2) para obtener \( Q\dot{z} = AQz+Bu \) y asu obtener el sistema
\[
    (3)
    \left\{
        \begin{array}{lll}
            \dot{z}(t) =
                \overbrace{Q^{-1}AQz}^{\tilde{A}}+ 
                \overbrace{Q^{-1}Bu}^{\tilde{B}}\\
            y(t) = \underbrace{
                CQz
            }_{\tilde{C}}
        \end{array}
    \right.
\]

Se obtiene la matriz de observabilidad para los sistemas (2) y (3) respectivamente.
\[
    \begin{split}
        (4)\;\;
        O & =\left[
        \begin{array}{cc}
            C \\
            CA \\
            CA^{2} \\
            \vdots \\
            CA^{n-1}
        \end{array}\right]
        =[C, CA, CA^{2}, \ldots, CA^{n-1}]^{T}
        \\
        (5)\;\;
        \tilde{O} & = \left[
        \begin{array}{cc}
            \tilde{C} \\
            \tilde{C}\tilde{A} \\
            \tilde{C}\tilde{A}^{2} \\
            \vdots \\
            \tilde{C}\tilde{A}^{n-1}
        \end{array}\right]
        =[\tilde{C}, \tilde{C}\tilde{A}, \tilde{C}\tilde{A}^{2}, \ldots, \tilde{C}\tilde{A}^{n-1}]^{T}
    \end{split}
\]

Debido a que \( \tilde{A}^{n} = Q^{-1}\tilde{A}^{n}Q \)
\[
    \begin{split}
        \tilde{O} & = [CQ, CQQ^{-1}AQ, CQQ^{-1}A^{2}Q, \ldots, CQQ^{-1}A^{n-1}Q]^{T}\\
        \tilde{O} & = [CQ, CAQ, CA^{2}Q, \ldots, CA^{n-1}Q]^{T}\\
        \tilde{O} & = [C, CA, CA^{2}, \ldots, CA^{n-1}]^{T}Q\\
        \tilde{O} & = OQ 
    \end{split}
\]

Lo que se busca es la matriz Q, por lo tanto, de la ecuación anterior se puede despejar de la siguiente forma
\[
    \begin{split}
        \tilde{O} & = OQ \\
        \tilde{O}Q^{-1} & = OQQ^{-1} \\
        \tilde{O}Q^{-1} & = OI \\
        \tilde{O}^{-1}\tilde{O}Q^{-1} & = \tilde{O}^{-1}O\\
        IQ^{-1} & = \tilde{O}^{-1}O\\
        Q^{-1} & = \tilde{O}^{-1}O \;\;(6)
    \end{split}
\]

Debido a que \( Q \) es invertible, \( rango(\tilde{O}) = rango(O)\) 

Tomando el sistema de entradas salidas
\[
    y(s)s^{n} + a_{1}y(s)s^{n-1} + \ldots + a_{n}y(s) = b_{1}u(s)s^{n-1} + \ldots + b_{n}u(s) \;\;(7)
\]

Se reacomoda de la siguiente forma
\[
    s^{n}y(s) + s^{n-1}(a_{1}y(s)-b_{1}u(s)) + \ldots + s(a_{n-1}y(s)-b_{n-1}u(s)) = b_{n}u(s)-a_{n}y(s) 
\]

Se considera la variable de estado \( X_{n}(s) = y(s) \), es decir
\[
    y(s) = [0,0,0,\ldots,1]X(s)
\]

Entonces se comienza a definir las variables de estado a partir de (7), de la siguiente forma
\[
    \begin{split}
        s^{n}y(s) + s^{n-1}(a_{1}y(s)-b_{1}u(s)) 
        + \ldots + s(a_{n-1}y(s)-b_{n-1}u(s)) & = b_{n}u(s)-a_{n}y(s)\\
        s^{n-1}sy(s) + s^{n-1}(a_{1}y(s)-b_{1}u(s)) 
        + \ldots + s(a_{n-1}y(s)-b_{n-1}u(s)) & = b_{n}u(s)-a_{n}y(s)\\
        s^{n-1}(sy(s)+a_{1}y(s)-b_{1}u(s))+ \ldots + s(a_{n-1}y(s)-b_{n-1}u(s)) & = b_{n}u(s)-a_{n}y(s)\\
        s^{n-1}
        \overbrace{(sy(s)+a_{1}y(s)-b_{1}u(s))}^{X_{n-1}(s)} + \ldots + s(a_{n-1}y(s)-b_{n-1}u(s)) & = b_{n}u(s)-a_{n}y(s)\\
        s^{n-1}X_{n-1}(s)+s^{n-2}(a_{2}y(s)-b_{2}u(s)) + \ldots + s(a_{n-1}y(s)-b_{n-1}u(s)) & = b_{n}u(s)-a_{n}y(s)\\
        s^{n-2}sX_{n-1}(s)+s^{n-2}(a_{2}y(s)-b_{2}u(s)) + \ldots + s(a_{n-1}y(s)-b_{n-1}u(s)) & = b_{n}u(s)-a_{n}y(s)\\
        s^{n-2}(sX_{n-1}(s)+(a_{2}y(s)-b_{2}u(s))) + \ldots + s(a_{n-1}y(s)-b_{n-1}u(s)) & = b_{n}u(s)-a_{n}y(s)\\
        s^{n-2}
        \overbrace{(sX_{n-1}+a_{2}y(s)-b_{2}u(s))}^{X_{n-2}(s)} + \ldots + s(a_{n-1}y(s)-b_{n-1}u(s)) & = b_{n}u(s)-a_{n}y(s)\\
    \end{split}
\]
\[
    \;\vdots
\]
\[
    \begin{split}
        s^{2}X_{2}(s)+s(a_{n-1}y(s)-b_{n-1}u(s)) & = b_{n}u(s)-a_{n}y(s)\\
        s
        \overbrace{(sX_{2}(s)+a_{n-1}y(s)-b_{n-1}u(s))}^{X_{1}(s)} & = b_{n}u(s)-a_{n}y(s)\\
        sX_{1}(s) & = b_{n}u(s)-a_{n}y(s)\\
    \end{split}
\]

Por lo tanto las variables de estado se definen como
\[
    \begin{split}
        X_{1}(s) & = sX_{2}(s)+a_{n-1}y(s)-b_{n-1}u(s)\\
        &\vdots\\
        X_{n-2}(s) & = sX_{n-1}(s)+a_{2}y(s)-b_{2}u(s)\\
        X_{n-1}(s) & = sX_{n}(s)+a_{1}y(s)-b_{1}u(s)\\
        X_{n}(s) & = y(s)
    \end{split}
\]

Del proceso para obtener las variables de estado se puede ver que 
\[
    \begin{split}
        sX_{1}(s) & = -a_{n}y(s)+b_{n}u(s)\\
        s^{2}X_{2}(s)+s(a_{n-1}y(s)-b_{n-1}u(s)) & = -a_{n}y(s)+b_{n}u(s)\\
        s^{2}X_{2}(s)+s(a_{n-1}y(s)-b_{n-1}u(s)) & = sX_{1}(s)\\
        sX_{2}(s)+(a_{n-1}y(s)-b_{n-1}u(s)) & = X_{1}(s)\\
        sX_{2}(s) & = X_{1}(s)-a_{n-1}y(s)+b_{n-1}u(s)
    \end{split}
\]

entonces la derivada de las variables de estado se pueden escribir de la siguiente forma
\[
    \begin{split}
        sX_{1}(s) & = -a_{n}y(s)+b_{n}u(s)\\
        sX_{2}(s) & = X_{1}(s)-a_{n-1}y(s)+b_{n-1}u(s)\\
        \vdots\\
        sX_{n-1}(s) & = X_{n-2}(s)-a_{2}y(s)+b_{2}u(s)\\
        sX_{n}(s) & = X_{n-1}(s)-a_{1}y(s)+b_{1}u(s)
    \end{split}
\]

Recordando que \( X_{n}(s) = y(s) \) se tiene que 
\[
    \begin{split}
        sX_{1}(s) & = -a_{n}X_{n}(s)+b_{n}u(s)\\
        sX_{2}(s) & = X_{1}(s)-a_{n-1}X_{n}(s)+b_{n-1}u(s)\\
        \vdots\\
        sX_{n-1}(s) & = X_{n-2}(s)-a_{2}X_{n}(s)+b_{2}u(s)\\
        sX_{n}(s) & = X_{n-1}(s)-a_{1}X_{n}(s)+b_{1}u(s)
    \end{split}
\]

Entonces es posible escribir el espacio de estados como
\[
    \begin{bmatrix}
        SX_{1}(s) \\
        SX_{2}(s) \\
        \vdots \\
        SX_{n}(s)
    \end{bmatrix}
    =
    \underbrace{
        \begin{bmatrix}
        0 & 0 & \cdots & -a_{n}\\
        1 & 0 & \cdots & -a_{n-1} \\
        \vdots & \ddots & \cdots & \vdots \\
        0 & \cdots & 1 & -a_{1}
        \end{bmatrix}
                }_{\tilde{A}}
    \begin{bmatrix}
        X_{1}(s) \\
        X_{2}(s) \\
        \vdots \\
        X_{n}(s)
    \end{bmatrix}
    +
    \underbrace{
        \begin{bmatrix}
        b_{n}\\
        b_{n-1}\\
        \vdots\\
        b_{1}
        \end{bmatrix}
                }_{\tilde{B}}
     u(s)
\]
\[
    y(s) = \underbrace{[0,0,\ldots,1]}_{\tilde{C}}X(s)
\]

Según la definición de la matriz similar de observabilidad
\[
    \tilde{O} = [
    \tilde{C},
    \tilde{C}\tilde{A},
    \tilde{C}\tilde{A}^{2},
    \cdots,
    \tilde{C}\tilde{A}^{n-1} ]^{T}
\]

Se usará una matriz de 3*3 para encontrar un patrón
\[
    \tilde{O} = 
    \begin{bmatrix}
        0 & 0 & 1 \\
        0 & 1 & -a_{1} \\
        0 & -a_{1} & -a_{2}+a_{1}^{2}
    \end{bmatrix}
\]

La inversa esta dada por
\[
    \tilde{C}^{-1} = 
    \begin{bmatrix}
        a_{2} & a_{1} & 1 \\
        a_{1} & 1 & 0 \\
        1 & 0 & 0
    \end{bmatrix}
\]

Por lo tanto, la inversa de la matriz similar de controlabilidad se puede expresar como 
\[
    \tilde{O}^{-1} = 
    \begin{bmatrix}
        a_{n-1} & a_{n-2} & \cdots & a_{1} & 1 \\
        a_{n-2} & a_{n-3} & \cdots & 1 & 0\\
        \vdots & \rotatebox{90}{\(\ddots\)} & \rotatebox{90}{\(\ddots\)} & \vdots & \vdots\\
        a_{1} & 1 & 0 & \cdots & 0\\
        1 & 0 & 0 & \cdots & 0
    \end{bmatrix}
\]

Considerando la matriz de observabilidad 
\[
    O = [C, CA, CA^{2}, \ldots, CA^{n-1}]^{T}
\]

La ecuación (6) se puede escribir como
\[
    Q^{-1} =
    \begin{bmatrix}
    a_{n-1} & a_{n-2} & \cdots & a_{1} & 1 \\
    a_{n-2} & a_{n-3} & \cdots & 1 & 0\\
    \vdots & \rotatebox{90}{\(\ddots\)} & \rotatebox{90}{\(\ddots\)} & \vdots & \vdots\\
    a_{1} & 1 & 0 & \cdots & 0\\
    1 & 0 & 0 & \cdots & 0
    \end{bmatrix}
    \begin{bmatrix}
        C\\
        CA\\
        CA^{2}\\
        \vdots\\
        CA^{n-1}
    \end{bmatrix}
\]