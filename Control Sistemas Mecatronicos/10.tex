\subsection{Ubiación de polos forma canonica controlable}

Considere el sistema SISO

\[
    (1)
    \left\{
        \begin{array}{lll}
            \dot{x}(t) = Ax(t) + Bu(t)\\
            y(t) = Cx(t)
        \end{array}
    \right.
\]

Se asume que el sistema (1) es controlable, y el polinomio caracteristico del sistema es 
\[
    \begin{split}
        P(s) & = det(sI-A) \\
        & = (s-q_{1}) ( s-q_{2}) \ldots (s-q_{n})\\
        & = s^{n} + a_{1}s^{n-1} + a_{2}s^{n-2} + \ldots + a_{n}
    \end{split}
\]

Haciendo \( z=Px \) o bien \( x=p^{-1}z \;\; (2)\) donde 
\[
    P^{-1} =
    [B, AB, A^{2}B, \cdots, A^{n-1}B ]
    \begin{bmatrix}
        1 & a_{1} & a_{2} & \cdots & a_{n-1} \\
        0 & 1 & a_{1} & \cdots & a_{n-2}\\
        \vdots & \ddots & \ddots & \ddots & \vdots\\
        0 & 0 & 0 & \ddots & a_{1}\\
        0 & 0 & 0 & \cdots & 1
    \end{bmatrix}
\]

Sustituyendo (2) en (1) se tiene que
\[
    \begin{split}
        p^{-1}\dot{z} & = AP^{-1}z + Bu\\
        (3)\;\;\dot{z} & = 
        \underbrace{PAP^{-1}}_{\tilde{A}}
        z + 
        \underbrace{PB}_{\tilde{B}}u \\
        \dot{z} & = 
        \begin{bmatrix}
            -a_{1} & -a_{2} & \cdots & -a_{n} \\
            1 & 0 & \cdots & 0 \\
            \vdots & \ddots & \cdots & \vdots \\
            0 & \cdots & 1 & 0
        \end{bmatrix} z
        +
        \begin{bmatrix}
            1\\ 0\\ \vdots \\ 0
        \end{bmatrix} u
    \end{split}
\]

Sustituyendo (2) en la retroalimentación de estado
\[
    \begin{split}
       (4) \;\;u & = r - kx\\
        & = r - kP^{-1}z\\
        & = r - \tilde{k}z
    \end{split}
\]

Donde \( \tilde{k}=kP^{-1} \;\; (5) \)

Sustituyendo (4) en (3)
\[
    \begin{split}
        \dot{z} & = PAP^{-1}z + PB(r-kP^{-1}z) \\
        & = PAP^{-1}z + PBr - PBkP^{-1}z \\
      (6) \;\;  & = P(A-Bk)P^{-1}z + PBr
    \end{split}
\]

Ya que \( A-Bk \) y \(P(A-Bk)P^{-1}\) son similares, se tiene que
\[
    \begin{split}
        P_{LC} & = det(sI-(A-Bk)) = det(sI-P(A-Bk)P^{-1})\\
        & = (s-\mu_{1}) (s-\mu_{2}) \ldots (s-\mu_{n})\\
        & = s^{n} +\tilde{a}_{1}s^{n-1} + \tilde{a}_{2}s^{n-2} + \ldots + \tilde{a}_{n}
    \end{split}
\]

Donde \( \mu_{1}, \mu_{2}, \ldots, \mu_{n} \) son los polos que se quieren ubicar, es decir, son conocidos, asi como \( \tilde{a}_{1},\tilde{a}_{2}, \ldots, \tilde{a}_{n} \)

Si \( \tilde{A} \) se obtiene del polinomio caracteristico del sistema \( A \), entonces se puede escribir el sistema similar en lazo cerrado \( \tilde{A}-\tilde{B}\tilde{k} \) de la siguiente forma
\[
    \tilde{A}-\tilde{B}\tilde{k}=
    \begin{bmatrix}
        -\tilde{a}_{1} & -\tilde{a}_{2} & \cdots & -\tilde{a}_{n} \\
        1 & 0 & \cdots & 0 \\
        \vdots & \ddots & \ddots & 0 \\
        0 & \cdots & 0 & 1
    \end{bmatrix} = \tilde{A}^{'}
\]

El problema consiste en despejar las ganancias \( \tilde{k}_{1}, \tilde{k}_{2}, \ldots, \tilde{k}_{n}  \)
\[
    \begin{split}
        \tilde{A} - \tilde{B} \tilde{k} & = \tilde{A}^{'} \\ 
    -\tilde{B} \tilde{k} & = \tilde{A}^{'} - \tilde{A} \\
    \tilde{B} \tilde{k} & = \tilde{A} - \tilde{A}^{'}\\
    \begin{bmatrix}
        1 \\ 0 \\ \vdots \\ 0
    \end{bmatrix}
    \begin{bmatrix}
        \tilde{k}_{1} & \tilde{k}_{2} & \ldots & \tilde{k}_{n}
    \end{bmatrix} & =
    \begin{bmatrix}
        -a_{1} & -a_{2} & \cdots & -a_{n} \\
        1 & 0 & \cdots & 0 \\
        \vdots & \ddots & \ddots & 0 \\
        0 & \cdots & 0 & 1
    \end{bmatrix}
    +
    \begin{bmatrix}
        \tilde{a}_{1} & \tilde{a}_{2} & \cdots & \tilde{a}_{n} \\
        -1 & 0 & \cdots & 0 \\
        \vdots & \ddots & \ddots & 0 \\
        0 & \cdots & 0 & -1
    \end{bmatrix} \\
    \begin{bmatrix}
        \tilde{k}_{1} & \tilde{k}_{2} & \ldots & \tilde{k}_{n} \\
        0 & 0 & \cdots & 0 \\
        \vdots & \ddots & \ddots & 0 \\
        0 & \cdots & 0 & 0
    \end{bmatrix} 
    & =
    \begin{bmatrix}
        \tilde{a}_{1} - a_{1} & \tilde{a}_{2} - a_{2} & \cdots & \tilde{a}_{n} - a_{n} \\
        0 & 0 & \cdots & 0 \\
        \vdots & \ddots & \ddots & 0 \\
        0 & \cdots & 0 & 0
    \end{bmatrix} \\
    \end{split}
\]

El vector de ganacias similares se puede escribir como
\[
    \begin{bmatrix}
        \tilde{k}_{1} \\ \tilde{k}_{2} \\ \vdots \\ \tilde{k}_{n}
    \end{bmatrix}
    =
    \begin{bmatrix}
        \tilde{a}_1 - a_{1} \\
        \tilde{a}_{2} -a_{1} \\
        \vdots \\
        \tilde{a}_n -a_{n}
    \end{bmatrix}
\]

De la ecuación (5) se puede despajar las ganancias en lazo cerrado
\[
    \begin{split}
        \tilde{k} & = kP^{-1} \\
        \tilde{k}P & = k \\
        k & =
        \begin{bmatrix}
            \tilde{a}_1 - a_{1} \\
            \tilde{a}_{2} -a_{1} \\
            \vdots \\
            \tilde{a}_{n-1} -a_{n-1}\\
            \tilde{a}_{n} -a_{n}
        \end{bmatrix}
        [B, AB, A^{2}B, \cdots, A^{n-1}B ]
        \begin{bmatrix}
            1 & a_{1} & a_{2} & \cdots & a_{n-1} \\
            0 & 1 & a_{1} & \cdots & a_{n-2}\\
            \vdots & \ddots & \ddots & \ddots & \vdots\\
            0 & 0 & 0 & \ddots & a_{1}\\
            0 & 0 & 0 & \cdots & 1
        \end{bmatrix}
    \end{split}
\]