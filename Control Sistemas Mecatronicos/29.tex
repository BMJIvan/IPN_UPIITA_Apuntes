\subsection{Forma canónica diagonal}

Considere el sistema (1)
\[
    \begin{split}
        (1) \;\; & y(k) + a_{1}y(k-1) + a_{2}y(k-2) + \ldots + a_{n}y(k-n) \\
            & = b_{1}u(k-n) + b_{2}u(k-2) + \ldots + b_{n}u(k-n)
    \end{split}
\]

Se obtiene la función de transferencia 
\[
    \begin{split}
        \frac{Y(z)}{U(z)} & = \frac{b_{1}z^{-1} + b_{2}z^{-2} + \ldots + b_{n}z^{-n}}{1 + a_{1}z^{-1} + a_{2}z^{-2} + \ldots + a_{n}z^{-n}} \frac{z^{n}}{z^{n}}\\ \;\;
        & = \frac{b_{1}z^{n-1} + b_{2}z^{n-2} + \ldots + b_{n}}{z^{n} + a_{1}z^{n-1} + a_{2}z^{n-2} + \ldots + a_{n}} \\
        & = \frac{b_{1}z^{n-1} + b_{2}z^{n-2} + \ldots + b_{n}}{(s-q_{1})(s-q_{2}) \ldots (s-q_{n})}
    \end{split}
\]

Sea \( q_{1}, q_{2}, \ldots, q_{n} \), raíces del polinomio P(z). Se asume que todas las raíces son distintas, entonces se define (2) en fracciones parciales
\[
    (3) \;\;
    \frac{C_{1}}{z-q_{1}}U(z) + \frac{C_{2}}{z-q_{2}}U(z) + \ldots + \frac{C_{n}}{z-q_{n}}U(z)
\]

Se definen las variables de estado de (3)
\[(4)
    \left\{
        \begin{array}{lll}
            X_{1}(z) = \frac{1}{z-q_{1}}U(z) \\ 
            x_{2}(z) = \frac{1}{z-q_{2}}U(z) \\
            \;\;\;\;\;\;\;\;\;\; \vdots \\
            X_{n}(z) = \frac{1}{z-q_{n}}
        \end{array}
    \right.
\]

Sustituyendo (4) en (3)
\[
    \begin{split}
        Y(z) & = X_{1}C_{1} + X_{2}C_{2} + \ldots + X_{n}C_{n} \\
        (5) \;\; Y(z) & = 
        \begin{bmatrix}
            C_{1} & C_{2} & \ldots & C_{n} 
        \end{bmatrix} X(z)
    \end{split}
\]

Si se toma una variable de estado
\[
    \begin{split}
        X_{1}(z) & = \frac{1}{z-q_{1}}U(z) \\
        X_{1}(z)(z-q_{1}) & = U(z) \\
        zX_{1}(z) - X_{1}(z)q_{1} & = U(z) \\
        zX_{1}(z) & = U(z) + X_{1}(z)q_{1}
    \end{split}
\]

Aplicando el mismo procedimiento a todas las variables de estado
\[(6)
    \left\{
        \begin{array}{lll}
            zX_{1}(z) & = U(z) + X_{1}(z)q_{1} \\ 
            zX_{2}(z) & = U(z) + X_{2}(z)q_{2} \\
            \;\;\;\;\;\;\;\;\;\; \vdots \\
            zX_{n}(z) & = U(z) + X_{n}(z)q_{n}
        \end{array}
    \right.
\]

Aplicando transformada inversa a (5) y (6)
\[(7)
    \left\{
        \begin{array}{lll}
            X_{1}(k+1) & = q_{1}X_{1}(k) + u(k) \\ 
            X_{2}(k+1) & = q_{2}X_{2}(k) + u(k) \\
            \;\;\;\;\;\;\;\;\;\; \vdots \\
            X_{n}(k+1) & = q_{n}X_{n}(k) + u(k) \\
            y(k) & = 
            \begin{bmatrix}
                C_{1} & C_{2} & \ldots & C_{n} 
            \end{bmatrix} X(k)
        \end{array}
    \right.
\]

A partir de (7) se obtiene la forma canónica diagonal
\[
    \begin{split}
        X(k+1) & = 
        \begin{bmatrix}
            q_{1} & 0 & \cdots & 0 \\
            0 & q_{2} & \cdots & 0 \\
            \vdots & \ddots & \ddots & \vdots \\
            0 & 0 & \cdots & q_{n}
        \end{bmatrix}X(k) +
        \begin{bmatrix}
            1 \\ 1 \\ \vdots \\ 1
        \end{bmatrix}u(k) \\
        y(k) & = 
        \begin{bmatrix}
            C_{1} & C_{2} & \ldots & C_{n}   
        \end{bmatrix} X(k)
    \end{split}
\]