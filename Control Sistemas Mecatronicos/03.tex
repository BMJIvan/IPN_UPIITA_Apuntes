\section{Controlabilidad y observabilidad de sistemas lineales}

Sea el sistema 

\[
    (1)
    \left\{
        \begin{array}{lll}
            \dot{x}(t) = \overbrace{A}^{\text{estabilidad}}x(t) + \underbrace{B}_{\textbf{controlabilidad}}u(t)\\
            y(t) = Cx(t)
        \end{array}
    \right.
\]

donde 

\[
    \begin{split}
        X & : n\times 1\\
        A & : n\times n\\
        B & : n\times m \;\; entradas\\
        u & : m\times 1\\
        C & : p\times n\\
        y & : P\times 1 \;\;salidas
    \end{split}
\]

Controlabilidad: Existencia de una entrada \(u(t)\) tal que cada variable de estado se pueda manipular de manera independiente. Es decir, las entradas cambian las variables.

Observabilidad: Consiste en determinar el estado inicial a partit de la salida \(y(t)\). Es decir, las condiciones iniciales afectan la salida.

Definición 1. El sistema (1) es controlable si existe \(u(t)\) tal que para todo estado inicial \(x_{0}=x(0)\) y todo estado final \(x_{f}=x(T)\), el sistema puede llevarse de \(x_{0}\) a \(x_{f}\) en tiempo finito.