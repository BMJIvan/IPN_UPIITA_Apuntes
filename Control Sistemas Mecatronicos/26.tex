\section{UNIDAD III Introducción a los sistemas discretos}

Un sistema en tiempo discreto se define en función de un periodo de muestreo \( T \) y un instante de muestreo \( k \)
\[
    x((x+1)k) = f(k,x(k), T)
\]

Si \( T \) es constante
\[
    x(k+1) = f(k, x(k))
\]

Las raíces del polinomio característico define la estabilidad del sistema.

Si todas las raíces tienen magnitud menor o igual que 1, el sistema es estable
\[
    |\mu_{i}| \leq 1
\]