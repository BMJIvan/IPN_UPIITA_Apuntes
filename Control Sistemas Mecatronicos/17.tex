\subsection{Ecuación de Lyapunov}

Sea \( F \) una matriz con los valores caracteristicos iguales a la dinamica deseada del observador

Se aplica una tranformación de similitud \( P \) tal que 
\[
    A-LC = PFP^{-1}
\]

Se desconocen \( P \) y \( L \), entonces
\[
    \begin{split}
        A-LC & = P^{-1}FP \\
        P(A-LC) & = FP \\
        PA - \underbrace{PL}_{\Tilde{L}}C & = FP\\
        -\tilde{L}C & = FP-PA\\
        \tilde{L}C & = PA-FP\\
    \end{split}
\]

\subsubsection{Procedimiento para proponer \( \tilde{L} \)}

\begin{enumerate}
    \item Construir una matriz F. Se recomienda que sea diagonal por bloques
    \item Seleccionar \(\tilde{L}\), tal que el par \( (F, \tilde{L}) \) sea controlaable, es decir
        \[
            C = \begin{bmatrix}
                    \tilde{L} &
                    F\tilde{k} &
                    F^{2}\tilde{L} &
                    \cdots &
                    F^{n}\tilde{L}
                \end{bmatrix}, \;\;  det(C) \not= 0
        \]
    \item La ecuación de Lyapunov es un sistema de ecuaciones lineales, en MATLAB se puede solucionar usando \( P=lyap(-F, A, -\tilde{L}C) \)
    \item Recordando que \( \tilde{L} = PL \), las ganancias se obtienen con 
    \[
            L = P^{-1}\tilde{L}
    \]
\end{enumerate}